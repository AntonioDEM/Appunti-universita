\chapter{Grandezze fisiche e unità di misura}
In fisica, una grandezza è la proprietà di un fenomeno, corpo o sostanza, che può essere espressa quantitativamente mediante un numero e un riferimento (ovvero che può essere misurata quantitativamente). by \href{https://it.wikipedia.org/wiki/Grandezza_fisica}{Wikipedia}
\begin{table}[!h]
	\centering
	\begin{tabular}{lll}
		\texttt{Grandezza}&\texttt{Nome}&\texttt{Simbolo}\\\hline
		Tempo&secondo&Simbolo\\
		Lunghezza&metro&m\\
		Quantità di materiale&mole&mol\\
		Temperatura termodinamica&kelvin&K\\
		Corrente elettrica&ampere&A\\
		Intensità luminosa&candela&cd\\\hline
	\end{tabular}
\caption{Unità fondamentali del sistema internazionale}
\label{table:1}
\end{table}\\
Per una questione di comodità di lettura esistono i multipli delle unità di
misura e vengono indicati con dei prefissi che consente di risurre il numero di
cifre, rendere più veloce la lettura e la scrittura.
\begin{table}[!h]
	\centering
	\begin{tabular}{cll|cll}
		\texttt{Fattore}&\texttt{Prefisso}&\texttt{Simbolo}&\texttt{Fattore}&\texttt{Prefisso}&\texttt{Simbolo}\\\hline
		$10^{18}$&exa-&E&$10^{-1}$&deci-&d\\
		$10^{15}$&peta-&P&$10^{-2}$&centi-&c\\
		$10^{12}$&tera-&T&$10^{-3}$&milli-&m\\
		$10^{9}$&giga-&G&$10^{-6}$&micro-&$\mu$\\
		$10^{6}$&mega-&M&$10^{-9}$&nano-&$n$\\
		$10^{3}$&kilo-&k&$10^{-12}$&pico-&$p$\\
		$10^{2}$&etto-&h&$10^{-15}$&femto-&$f$\\
		$10^{1}$&deca-&da&$10^{-18}$&atto-&$a$\\\hline

	\end{tabular}
\caption{Prefissi per le unità SI$^a$}
\end{table}
\section{Sistema internazionale delle unità di misura}
l sistema internazionale di unità di misura (in francese: \textit{Système international d'unités}), abbreviato in S.I. (pronunciato esse-i
), è il più diffuso sistema di unità di misura. Nei paesi anglosassoni sono ancora impiegate delle unità consuetudinarie, un esempio sono quelle statunitensi.
La difficoltà culturale nel passaggio della popolazione da un sistema all'altro è essenzialmente legato a radici storiche. Il sistema internazionale impiega per la maggior parte unità del sistema metrico decimale nate nel contesto della rivoluzione francese: le unità S.I. hanno gli stessi nomi e praticamente la stessa grandezza pratica delle unità metriche. Il sistema è un sistema tempo-lunghezza massa che è stato inizialmente chiamato Sistema MKS, per distinguerlo dal similare Sistema CGS. Le sue unità di misura erano infatti metro, chilogrammo e secondo invece che centimetro, grammo, secondo. By \href{https://it.wikipedia.org/wiki/Sistema_internazionale_di_unità_di_misura}{Wikipedia}
