\documentclass{book}

\usepackage[utf8]{inputenc}
\usepackage{titlesec}
\usepackage{easylist}
\usepackage{hanging}
\usepackage{hyperref}
\usepackage[a4paper,top=2.0cm,bottom=2.0cm,left=2.0cm,right=3.0cm]{geometry}
\usepackage{blindtext}
\usepackage{tipa}
\usepackage{epigraph}
\usepackage{enumerate}
\usepackage{longtable}
\usepackage{setspace}
\usepackage{verbatim}
\usepackage[T1]{fontenc}
\usepackage{graphicx}
\usepackage[italian]{babel}
\usepackage{amsmath}
\usepackage{pbox}
\usepackage{fancyhdr}
\usepackage{cancel}
\usepackage{tabularx}
\usepackage{booktabs}
\usepackage{multirow}
\usepackage{longtable}
\usepackage{tikz}
\usepackage{tikz-qtree}
\usepackage{subfig}
\usepackage{xcolor}
\usepackage{amssymb}
\usepackage{mathrsfs}
\usepackage{textcomp}
\usepackage{circuitikz}
\usepackage{pifont}

\usepackage{listings}
\usepackage{color}

\definecolor{mygreen}{rgb}{0,0.6,0}
\definecolor{mygray}{rgb}{0.5,0.5,0.5}
\definecolor{mymauve}{rgb}{0.58,0,0.82}

\lstset{ 
  backgroundcolor=\color{white},   % choose the background color; you must add \usepackage{color} or \usepackage{xcolor}; should come as last argument
  basicstyle=\footnotesize,        % the size of the fonts that are used for the code
  breakatwhitespace=false,         % sets if automatic breaks should only happen at whitespace
  breaklines=true,                 % sets automatic line breaking
  captionpos=b,                    % sets the caption-position to bottom
  commentstyle=\color{mygreen},    % comment style
  deletekeywords={...},            % if you want to delete keywords from the given language
  escapeinside={\%*}{*)},          % if you want to add LaTeX within your code
  extendedchars=true,              % lets you use non-ASCII characters; for 8-bits encodings only, does not work with UTF-8
  firstnumber=1000,                % start line enumeration with line 1000
  frame=single,	                   % adds a frame around the code
  keepspaces=true,                 % keeps spaces in text, useful for keeping indentation of code (possibly needs columns=flexible)
  keywordstyle=\color{blue},       % keyword style
  language=Octave,                 % the language of the code
  morekeywords={*,...},            % if you want to add more keywords to the set
  numbers=left,                    % where to put the line-numbers; possible values are (none, left, right)
  numbersep=5pt,                   % how far the line-numbers are from the code
  numberstyle=\tiny\color{mygray}, % the style that is used for the line-numbers
  rulecolor=\color{black},         % if not set, the frame-color may be changed on line-breaks within not-black text (e.g. comments (green here))
  showspaces=false,                % show spaces everywhere adding particular underscores; it overrides 'showstringspaces'
  showstringspaces=false,          % underline spaces within strings only
  showtabs=false,                  % show tabs within strings adding particular underscores
  stepnumber=2,                    % the step between two line-numbers. If it's 1, each line will be numbered
  stringstyle=\color{mymauve},     % string literal style
  tabsize=2,	                   % sets default tabsize to 2 spaces
  title=\lstname                   % show the filename of files included with \lstinputlisting; also try caption instead of title
}

\linespread{1.5} % l'interlinea

\frenchspacing

\newcommand{\abs}[1]{\lvert#1\rvert}

\usepackage{floatflt,epsfig}

\usepackage{multicol}
\newcommand\yellowbigsqcup[1][\displaystyle]{%
  \fboxrule0pt
  \ifx#1\textstyle\fboxsep-0.6pt\else\fboxsep-1.25pt\fi
  \mathrel{\fcolorbox{white}{yellow}{$#1\bigsqcup$}}}

\title{Appunti Fisica}
\author{Nicola Ferru}
\date{}
\begin{document}
\maketitle
\tableofcontents
\part{fisica 1}
\section{moto rettilineo uniformemente accelerato}
Moto rettilineo uniformemente accelerato. La definizione di moto rettilineo uniformemente accelerato è: il moto di un corpo con accelerazione costante lungo una traiettoria retta sempre nella stessa direzione e identico verso.
\begin{multicols}{3} 
$V_{S_0}=30,0m/s$\\
$V_F=5,00m/s$\\
$A_s=-2.00m/s$\\
$X_{F_0}=I_{SF}=155,5m$\\
$X_s(t)=X_{S_0}+X_{S_0}t+\frac{1}{2}A_st^2$\\
$X_s(t)=V_{S_0}+\frac{1}{2}A_st^2$\\
$X_F(t)=X_{F_0}+V_{F_0}t$
\end{multicols}
\begin{multicols}{2} 
\begin{tikzpicture}[domain=-2:7] 
    \draw[very thin, color=gray] (-2.1,-1.1) grid (6.9,3.9);
    \draw[->] (-2.2,0) -- (7.2,0) node[right] {$t$}; 
    \draw[->] (0,-1.2) -- (0,4.2) node[above] {$x$};
    \draw[color=red] (1,1.63) -- (1,0);
    \draw[color=blue] (5,2.9) -- (0,1.3);
    \draw (-0.55,-1) parabola[bend pos=0.5] bend +(0,3) +(6,0);
    \draw (-0.46,-1) parabola[bend pos=0.5, color=red] bend +(0,4) +(7,0);
    \filldraw (0,1.3) circle (2pt) node[align=left,   below] {$X_{f_0}$};
  \end{tikzpicture}\\
  $(x_f(t)-x_{f_0})=X_{f}(t_0)$\\
  $X_s(t_c)=X_f(t_0)$\\
  $V_st_e+\frac{1}{2}A_st^2c=X_{F_0}+V_{F_0}+V_{F_0}+V_{F_0}tc$
\end{multicols}
$\alpha{x^2}+\beta{x}+\gamma=0$\\
$x=\frac{-\beta\pm\sqrt{\beta-\gamma}}{2\alpha}$ $\Delta\geq 0$\\
$\tilde{x^2}+\tilde{2\beta x}+\gamma=0$\\
$x=\sqrt{\tilde{\beta}}$\\
$\frac{1}{2}(V_{s_0}-V_{F_0})T_c-X_{F0}=0$\\
$t^2_c+\frac{2}{|A_s|}(V_{s_0}-V_{f_0})t_c-\frac{2}{A_s}X_{f_0}=0$\\
$A_s=-|A_s|$\\
$t_c=-[-\frac{I}{A_s}(V_{s_0}-V_{f_0})]\pm\sqrt{(v_{s_0}-v_{f_0})/A^2_s-\frac{2}{|A_s|}X_{f_0}}=156,25-155=1,25$\\
$t_{c_-}=12,5-1,00s=11.5s$
\subsection{Un problema d'esempio}
Si Lascia cadere un sasso in un pozzo. il tempo nell'acqua viene percepito con un ritardo di 7.40s, a quale distanza dall'imboccatura del pozzo si trova la superficie dell'acqua? La velocità del suono nell'aria è 336 m/s.
\begin{center}
	\begin{tikzpicture}[domain=-2:7] 
    		\filldraw (1.5,3.2) circle (2pt) node[align=left,   below] {pietra};
    		\draw[,color=black] (1,3) -- (1,0);
    		\draw[color=black] (3,3) -- (3,0);
		\draw[->,color=red] (0.5,3) -- (0.5,0);
	\end{tikzpicture}
\end{center}
\begin{multicols}{3} 
$V_s=336m/s$ $\Delta t_{tot}=4,40s$\\
$y(t)=y_0+V_0t+\frac{1}{2}at^2$\\
$y=0$ $y_0=0$ $V_0=0$ $a=-g$\\
$y(t)=h-\frac{1}{2}gt^2$\\
$\Delta t_{tot}=t_{caduta}+t_{suono}$\\
$h=V_s*t_{suono}$\\
$t_{suono}=h/V_s$\\
$y(t_c)=0$\\
$h-\frac{1}{2}gt^2_c=0$\\
$\Delta t_{tot}=\sqrt{\frac{2h}{g}}+\frac{h}{V_s}$\\
$\Delta{t_{tot}}=-\frac{h}{V_S}=\sqrt{\frac{2h}{g}}$\\
$\Delta{t_{tot}}-\frac{h}{V_s}>0$\\
$(\Delta{t_{tot}}-\frac{h}{V_s})^2=\frac{2h}{g}$\\
$\Delta{t^2_{tot}}+\frac{h^2}{V^2_s}-\frac{2h}{v+V_x}\Delta{t_{tot}}=\frac{2h}{g}$\\
$\frac{h^2}{V^2_s}-2(\frac{\Delta{t_{tot}}}{V_s}+\frac{I}{g})h+\Delta{t^2_{tot}}=0$\\
$h^2-2V^2_s(\frac{\Delta{t_{tot}}}{V_s}+\frac{I}{g})h+\Delta{t^2_{tot}}=0$\\
$h=V^2_s(\frac{\Delta{t_{tot}}}{V_s}+\frac{I}{g})h+V^2_s\Delta{t^2_{tot}}=0$\\
$h=V^2_s(\frac{\Delta{t_{tot}}}{V_s}+\frac{I}{g})\pm\sqrt{[\frac{\Delta{t_{tot}}}{V_s}+\frac{I}{g}]^2-\frac{2h}{v+V_x}\Delta{t_{tot}}}$
$\Delta{t_{tot}}-\frac{h}{V_s}>0$
\end{multicols}
\begin{tikzpicture}[domain=-2:6] 
    \draw[very thin,color=gray] (-2.1,-1.1) grid (5.9,3.9);
    \draw[->] (-2.2,0) -- (6.2,0) node[right] {$t$}; 
    \draw[->] (0,-1.2) -- (0,4.2) node[above] {$x$};
    \filldraw (-1,-0.4) circle (2pt) node[align=left,   below] {0};

    \draw [green, thick, domain=-2:2] plot (\x, {4-\x*\x}); 
  \end{tikzpicture}
\begin{multicols}{3}
	$x(t)=x_0+V_{x_0}t$\\
	$y(t)=y_0+V_{y_0}t$\\
	$x-x_0=V_{x_0}$\\
	$y-y_0=V_{y_0}$\\
	$\frac{y-y_0}{x-x_0}=\frac{V_{x_0}}{V_{y_0}}$\\
	$x(t)=x_0+V_{x_0}t+\frac{1}{2}a_xt^2$\\
	$y(t)=y_0+V_{y_0}t+\frac{1}{2}a_yt^2$\\
	$\frac{y-y_0}{x-x_0}=\frac{V_{y_0}}{V_{x_0}}=\frac{ay}{ax}$\\
	$\frac{1}{2}\frac{V_{y_0}}{g}=-\frac{1}{2}\frac{g}{V_{x^2_0}}\frac{V_{x^r_0}*V^2_{y_0}}{g^2}$\\
	$y-y_m=-\frac{1}{2}\frac{g}{}$
\end{multicols}
\begin{tikzpicture}[domain=-2:6] 
    \draw[very thin,color=gray] (-2.1,-1.1) grid (5.9,3.9);
    \draw[->] (-2.2,0) -- (6.2,0) node[right] {$t$}; 
    \draw[->] (0,-1.2) -- (0,4.2) node[above] {$x$};
    \filldraw (-1,-0.4) circle (2pt) node[align=left,   below] {0};

    \draw [green, thick, domain=-2:2] plot (\x, {4-\x*\x}); 
  \end{tikzpicture}
\begin{multicols}{3}
	$A_y=-g$\\
	$y(t)=y_0+V_{y_0}t+\frac{1}{2}A_yt^2$\\
	$y_0=0$ $x_0=0$\\
	$V(t)=V_{y_0}t-\frac{1}{2}gt^2$\\
	$x(t)=x_0+V_{y_0}t$\\
	$x(t)=V_{x_0}t$\\
	$\begin{cases}
		x(t)=V_{x_0}t\\
		y()=V_{y_0}t-\frac{1}{2}gt^2\\
	\end{cases}$\\
	$y-y_m=\alpha(x-x_m)^2$\\
	$V_y(t)=V_{y_0}-gt$
	$-V_m=\alpha x^2_m$\\
	$t=\frac{x}{V_{y_0}}$\\
	$y=V_{y_0}-\frac{1}{2}g\frac{x^2}{V_{x_0}^2}$\\
	$y-y_m=\alpha x^2+\alpha x^2_m-2\alpha x x_m$\\
	$\alpha=-\frac{1}{2}$\\
	$X_m=\frac{V_{x_0}*V_{y_0}}{g}$\\
	$t_m=\frac{V_{y_0}}{g}$\\
	$V_{y_0}-gt_m=0$\\
	$y_m=V_{y_0}\frac{V_{y_0}}{g}-\frac{1}{2}g\frac{V^2_{y_0}}{g^2}$\\
	$\frac{1}{2}\frac{V_{y_{0}^2}}{g}=-\frac{1}{2}\frac{g}{V_{x_0^2}}\frac{V_{x^r_0}}{g^2}$\\
	$y-y_m=-\frac{1}{2}\frac{g}{V_{x_{0}^2}}(x-x_m)^2$\\
\end{multicols}
\begin{tikzpicture}[domain=-2:6] 
    \draw[very thin,color=gray] (-2.1,-1.1) grid (5.9,3.9);
    \draw[->] (-2.2,0) -- (6.2,0) node[right] {$t$}; 
    \draw[->] (0,-1.2) -- (0,4.2) node[above] {$x$};
    \filldraw (-1,-0.4) circle (2pt) node[align=left,   below] {0};

    \draw [green, thick, domain=-2:2] plot (\x, {4-\x*\x}); 
\end{tikzpicture}
\begin{multicols}{3}
	$X_p=r*\cos{\sigma}$\\
	$\cos{\sigma}=\frac{x_p}{r}$\\
	$y_p=r\sin{\sigma}$\\
	$X^2_p+y_p^2=r^2$\\
	$\frac{y_p}{x_p}=\frac{\sin{\sigma}}{\cos{\sigma}}=\tan{\sigma}$\\
	$\cos{\sigma}=\cos{-\sigma}$\\
	$\sin{\sigma}=-\sin{-\sigma}$
\end{multicols}
\section{I vettori}
\subsection{Proiezione dei vettori prodotto scalare}
\begin{tikzpicture}[domain=-2:6] 
    \draw[very thin,color=gray] (-2.1,-1.1) grid (5.9,3.9);
    \draw[->] (-2.2,0) -- (6.2,0) node[right] {$t$}; 
    \draw[->] (0,-1.2) -- (0,4.2) node[above] {$x$};
    \filldraw (-1,-0.4) circle (2pt) node[align=left,   below] {0};

    \draw [green, thick, domain=-2:2] plot (\x, {4-\x*\x}); 
  \end{tikzpicture}\\
\begin{multicols}{3} 
$L*L=1$\\
$J*J=1$\\
$\overrightarrow{a}*\overrightarrow{i}=a_x$\\
$\overrightarrow{a}*$\\
$\overrightarrow{a}=\overrightarrow{a}_x\overrightarrow{I}+a_y\overrightarrow{J}$\\
$ax=\overrightarrow{a}*\overrightarrow{J}=||a||*||\overrightarrow{J}||\cos{\phi}=||\overrightarrow{a}||*\cos{\phi}$\\
$\overrightarrow{a}=a_x\overrightarrow{L}+a_y\overrightarrow{J}$\\
$\overrightarrow{b}=b_x\overrightarrow{L}+b_y\overrightarrow{J}$\\
$\overrightarrow{a}*\overrightarrow{b}=(a_x\overrightarrow{J}+a_y\overrightarrow{J})*(b_x\overrightarrow{J}+b_y\overrightarrow{J})$\\
$\overrightarrow{a}*\overrightarrow{b}=a_x*b_x+a_yb_y$\\
$||\overrightarrow{a}||=a_{x^*2}+a_{y^2}=\overrightarrow{a}*\overrightarrow{a}$\\
$\overrightarrow{r(t)}=\overrightarrow{r_0}+V_0t+\frac{1}{2}\overrightarrow{y}t^2$\\
$\overrightarrow{r}*\overrightarrow{J}=y=\overrightarrow{r}*\overrightarrow{J}+\overrightarrow{V_0}*\overrightarrow{J}$\\
$\cos{\frac{\pi}{2}*\phi}=\sin{\phi}$\\
$x=x_0+V_xt$\\
$y=y_0+V_0t-\frac{1}{2}gt^2$\\
\end{multicols} 
\subsubsection{Moto balistico}
$x=x_0+V_{0x}t$\\
$y=y_0+V_{0y}t-\frac{1}{2}gt^2$\\
$x=0$\\
$y=h$\\
$V_{0y}=\overrightarrow{V_0}*\overrightarrow{J}=||\overrightarrow{V}||*||\overrightarrow{J}||$\\
$h=\frac{1}{2}gt^2$

\subsection{Primitive di una funsione}
\begin{tikzpicture}[domain=-2:6] 
    \draw[very thin,color=gray] (-2.1,-1.1) grid (5.9,3.9);
    \draw[->] (-2.2,0) -- (6.2,0) node[right] {$t$}; 
    \draw[->] (0,-1.2) -- (0,4.2) node[above] {$x$};
    \filldraw (-1,-0.4) circle (2pt) node[align=left,   below] {0};

    \draw [green, thick, domain=-2:2] plot (\x, {4-\x*\x}); 
\end{tikzpicture}\\
$\frac{d}{ax}\mathcal{A}_x=f(x)$
\begin{multicols}{2}
$\mathcal{A}(x)=\int^?_? i\mathcal{A}(x)=\int^?_? f(x)dx=$integrale indefinita\\
$P(x)=\mathcal{A}(x)+c\to$costante arbitraria\\
$P(x_2)-P(x_1)=\mathcal{A}(x_2)+c-\mathcal{A}(x_1)-c=\mathcal{A}(x_2)-\mathcal{A}(x_1)$\\
\end{multicols}
\subsubsection{Integrali definito}
$\mathcal{A}(x_2)-\mathcal{A}(x_1)=\int^{\mathcal{A}(x_2)}_{\mathcal{A}(x_1)}d\mathcal{A}(x)=\int^{\mathcal{A}(x_2)}_{\mathcal{A}(x_1)}f(x)dx$\\
Teorema dell'energia cinetica $\overrightarrow{F}_R$ risultante delle forze.\\
$dL=\overrightarrow{F}_R*d\overrightarrow{r}$ lavoro elementare fonte della risultante.
\begin{multicols}{2}
	$L_{1,2}=\int^{\overrightarrow{r}_2}_{\overrightarrow{r}_1}F_R*d\overrightarrow{r}$\\ 
	$\overrightarrow{F}_R=m\overrightarrow{a}=m\frac{d\overrightarrow{v}}{dt}$\\
	$L_{1,2}=m\int^{\overrightarrow{r}_2}_{\overrightarrow{r}_1}\frac{d\overrightarrow{v}}{dt}*d\overrightarrow{r}=m\int^{\overrightarrow{r}_2}_{\overrightarrow{r}_1}d\overrightarrow{v}*\frac{d\overrightarrow{r}}{dt}=m\int^{\overrightarrow{r}_2}_{\overrightarrow{r}_1}d\overrightarrow{v}*\overrightarrow{v}$
	$\frac{d}{dt}v^2=\frac{d}{dt}\overrightarrow{v}*\overrightarrow{v}=m\int^{
		\overrightarrow{V}_2}_{\overrightarrow{V}_1}d\overrightarrow{V}*\overrightarrow{V}$\\
	$\frac{d}{dt}V^2=\frac{d}{dt}\overrightarrow{V}*\overrightarrow{V}=\frac{d}{dt}\overrightarrow{V}
	+\overrightarrow{V}\frac{d\overrightarrow{V}}{dt}=2\overrightarrow{V}*\frac{d\overrightarrow{V}}{dt}$
\end{multicols}
Derivata del prodotto di funzione $\sigma$
\begin{itemize}
	\item $\frac{d}{dt}(f(t)*g(t))=(\frac{d}{at}*f(t))g(r)+f(t)*\frac{d}{dt}g(t)$
	\item $\frac{dV^2}{dt}=2\overrightarrow{V}*\frac{d\overrightarrow{V}}{dt}$
	\item $dV^2=2\overrightarrow{V}*d\overrightarrow{V}$
		\begin{tabular}{|l|}
			\hline
				$\overrightarrow{V}*d\overrightarrow{V}=\frac{1}{2}dV^2$\\
			\hline
		\end{tabular}
\end{itemize}
$L_{1,2}=\frac{1}{2}m\int^{V^2_2}_{V^2_1}dV^2=\frac{1}{2}m(V^2_2-V^2_1)$\\
\begin{tabular}{|l|}
	\hline
		$L_{1,2}=\frac{1}{2}mV^2_2-\frac{1}{2}mV^2_1$
	\\\hline
\end{tabular}
$K=\frac{1}{2}mV^2$ energia cinetica\\
\subsubsection{esempi $\sigma$}
\subsection{Primitive di una funsione}
$ax=g\sin{\sigma}$ $F_r=m\overrightarrow{g}+\overrightarrow{F}n$\\
$F_{R_{y}}=0$ $F_{R_{x}}=ma_{x}=mg\sin\sigma$\\
\begin{tikzpicture}[domain=-2:6] 
    \draw[very thin,color=gray] (-2.1,-1.1) grid (5.9,3.9);
    \draw[->] (-2.2,0) -- (6.2,0) node[right] {$t$}; 
    \draw[->] (0,-1.2) -- (0,4.2) node[above] {$x$};
    \filldraw (-1,-0.4) circle (2pt) node[align=left,   below] {0};

    \draw [green, thick, domain=-2:2] plot (\x, {4-\x*\x}); 
\end{tikzpicture}\\


\end{document}
