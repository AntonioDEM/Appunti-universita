\documentclass{book}

\usepackage[utf8]{inputenc}
\usepackage{titlesec}
\usepackage{easylist}
\usepackage{hanging}
\usepackage{hyperref}
\usepackage[a4paper,top=2.0cm,bottom=2.0cm,left=2.0cm,right=3.0cm]{geometry}
\usepackage{blindtext}
\usepackage{tipa}
\usepackage{epigraph}
\usepackage{enumerate}
\usepackage{longtable}
\usepackage{setspace}
\usepackage{verbatim}
\usepackage[T1]{fontenc}
\usepackage{graphicx}
\usepackage[italian]{babel}
\usepackage{amsmath}
\usepackage{pbox}
\usepackage{fancyhdr}
\usepackage{cancel}
\usepackage{tabularx}
\usepackage{booktabs}
\usepackage{multirow}
\usepackage{longtable}
\usepackage{tikz}
\usepackage{tikz-qtree}
\usepackage{subfig}
\usepackage{xcolor}
\usepackage{amssymb}
\usepackage{amsmath}
\usepackage{mathrsfs}
\usepackage{textcomp}
\usepackage{circuitikz}
\usepackage{pifont}
\usepackage{imakeidx}
\usepackage{verbatim}
\usepackage{dsfont}
\usepackage{listings}
\usepackage{color}
\usepackage{upgreek}
\usepackage{tasks}
\usepackage{exsheets}
\usepackage{pgfplots}
\usepackage{amsthm}

\SetupExSheets[question]{type=exam}

\definecolor{mygreen}{rgb}{0,0.6,0}
\definecolor{mygray}{rgb}{0.5,0.5,0.5}
\definecolor{mymauve}{rgb}{0.58,0,0.82}

\lstset{ 
  backgroundcolor=\color{white},   % choose the background color; you must add \usepackage{color} or \usepackage{xcolor}; should come as last argument
  basicstyle=\footnotesize,        % the size of the fonts that are used for the code
  breakatwhitespace=false,         % sets if automatic breaks should only happen at whitespace
  breaklines=true,                 % sets automatic line breaking
  captionpos=b,                    % sets the caption-position to bottom
  commentstyle=\color{mygreen},    % comment style
  deletekeywords={...},            % if you want to delete keywords from the given language
  escapeinside={\%*}{*)},          % if you want to add LaTeX within your code
  extendedchars=true,              % lets you use non-ASCII characters; for 8-bits encodings only, does not work with UTF-8
  firstnumber=1000,                % start line enumeration with line 1000
  frame=single,	                   % adds a frame around the code
  keepspaces=true,                 % keeps spaces in text, useful for keeping indentation of code (possibly needs columns=flexible)
  keywordstyle=\color{blue},       % keyword style
  language=Octave,                 % the language of the code
  morekeywords={*,...},            % if you want to add more keywords to the set
  numbers=left,                    % where to put the line-numbers; possible values are (none, left, right)
  numbersep=5pt,                   % how far the line-numbers are from the code
  numberstyle=\tiny\color{mygray}, % the style that is used for the line-numbers
  rulecolor=\color{black},         % if not set, the frame-color may be changed on line-breaks within not-black text (e.g. comments (green here))
  showspaces=false,                % show spaces everywhere adding particular underscores; it overrides 'showstringspaces'
  showstringspaces=false,          % underline spaces within strings only
  showtabs=false,                  % show tabs within strings adding particular underscores
  stepnumber=2,                    % the step between two line-numbers. If it's 1, each line will be numbered
  stringstyle=\color{mymauve},     % string literal style
  tabsize=2,	                   % sets default tabsize to 2 spaces
  title=\lstname                   % show the filename of files included with \lstinputlisting; also try caption instead of title
}

\linespread{1.2} % l'interlinea

\frenchspacing

\newcommand{\abs}[1]{\lvert#1\rvert}

\usepackage{floatflt,epsfig}

\usepackage{multicol}
\newcommand\yellowbigsqcup[1][\displaystyle]{%
  \fboxrule0pt
  \ifx#1\textstyle\fboxsep-0.6pt\else\fboxsep-1.25pt\fi
  \mathrel{\fcolorbox{white}{yellow}{$#1\bigsqcup$}}}

\title{Appunti di Algebra e geometria}
\author{Nicola Ferru}
\date{}
\makeindex[columns=3, title=Alphabetical Index, intoc]

\begin{document}
\maketitle
\tableofcontents
\listoftables
\listoffigures
\section{Premesse\dots}
In questo repository sono disponibili pure le dimostrazioni grafiche realizzate
con \textit{Geogebra} consiglio a tutti di dargli un occhiata e di stare
attenti perché possono essere presenti delle modifiche per migliorare il
contenuto degli stessi appunti, comunque solitamente vengono fatte revisioni
tre/quattro volte alla settimana perché sono in piena fase di sviluppo. Ricordo
a tutti che questo è un progetto volontario e che per questo motivo ci
potrebbero essere dei rallentamenti per cause di ordine superiore e quindi
potrebbero esserci meno modifiche del solito oppure potrebbero esserci degli
errori, chiedo la cortesia a voi lettori di contattarmi per apportare una
modifica.
\begin{center}
	Cordiali saluti
\end{center}


\section{Simboli}
\begin{multicols}{3}
	$\in$ Appartiene\\
	$\notin$ Non appartiene\\
	$\exists$ Esiste\\
	$\exists !$ Esiste unico\\
	$\subset$ Contenuto strettamente\\
	$\subseteq$ Contenuto\\
	$\supset$ Contenuto strettamente\\
	$\supseteq$ Contiene\\
	$\Rightarrow$ Implica\\
	$\Longleftrightarrow$ Se e solo se\\
	$\neq$ Diverso\\
	$\forall$ Per ogni\\
	$\ni :$ Tale che\\
	$\leq$ Minore o uguale\\
	$\geq$ Maggiore o uguale\\
	$\alpha$ alfa\\
	$\beta$ beta\\
	$\gamma$ gamma\\
	$\Gamma$ Gamma\\
	$\delta,\Delta$ delta\\
	$\epsilon$ epsilon\\
	$\sigma,\Sigma$ sigma\\
	$\rho$ rho
\end{multicols}

\chapter{Vettori}
\section{Spazio Vettoriale}
\newtheorem{SpaVet}{Spazio Vettoriale}
\begin{SpaVet}
	Uno spazio vettoriale reale (R-spazio vettoriale) è un insieme \textit{V} in
	cui sono definite un'operazione di \texttt{SOMMA} tra elementi di
	\textit{V} e un'operazione di \texttt{Prodotto tra un reale} e un elemento
	di V che soddisfano 8 proprietà:
\end{SpaVet}
\begin{enumerate}
	\item La somma è associativa quando $\forall v_1, \text{ } v_2, \text{ } v_3 
		\in V$ $\left(v_1+v_2\right)+v_3=v_1+\left(v_2+v_3\right)$;
	\item La somma è commutativa quando $\forall v_1, v_2 \in V\text{ }
		v_1+v_2=v_2+v_1$
	\item Esistenza elemento neutro 0 se e solo se $\forall v\in V \text{ }
		v+0=0+v=v$
	\item Esistenza opposto $-v$ se e solo se $\forall v \in V \text{ }
		v+(-v)=(-v)+v=0$
	\item Il prodotto per uno scalare è assoluto quando $\forall c_1,c_2 \in
		R, \forall v\in V \text{ } c_1(c_2v)=(c_1c_2)v$
	\item Il prodotto per uno scalare è distributiva quando $\forall c_1,c_2 \in
		R, \forall v\in V \text{ } (c_1+c_2)v=c_1v+c_2v$
	\item Il prodotto per uno scalare è distributiva quando $\forall c \in
		R, \forall v_1, v_2\in V \text{ }c(v_1+v_2)=cv_1+cv_2$
	\item Esistenza elemento neutro 1 quando $\forall v\in V \text{ } 1v=v$
\end{enumerate}
\begin{description}
	\item[ES:] $V_0^2\text{ } V_0^3$
	\item[ES:] $f:\mathds{R}\to \mathds{R}\text{ } x^2, \text{ } g(x)=e^x, \text{ } f(x)+g(x)=x^2+e^x\text{ } 3f(x)=3x^2$
	\item[ES:] $\mathds{R}^n$ n-uple di numeri reali
	\[
	\begin{matrix}
		\begin{bmatrix}
			x_1\\
			x_2\\
			\vdots\\
			x_n
		\end{bmatrix}+\begin{bmatrix}
			y_1\\
			y_2\\
			\vdots\\
			y_n
		\end{bmatrix}=\begin{bmatrix}
			x_1+y_1\\
			x_2+y_2\\
			\vdots\\
			x_n+y_n
		\end{bmatrix}&C\in\mathds{R} \text{ } c\begin{bmatrix}
			x_1\\
			x_2\\
			\vdots\\
			x_n
		\end{bmatrix}=\begin{bmatrix}
			cx_1\\
			cx_2\\
			\vdots\\
			cx_n
		\end{bmatrix}
	\end{matrix}
	\]
	\item[ES:] $\mathds{R}_n[x]$ polinomi di grado $\leq n$ nella variabile $x$ a coefficiente reale
		\begin{itemize}
			\item $p(x)=a_0+a_1x+a_2x^2+\dots+a_nx^n$
			\item $q(x)=b_0+b_1x+b_2x^2+\dots+b_nx^n$
		\end{itemize}
	\item[ES:] $\mathds{R}[x]$ polinomio di grado qualsiasi
		\begin{equation*}
			\begin{matrix}
				p(x)+q(x)=a_0+b_0+(a_1+b_1)x+\dots+(a_n+b_n)x^n\\
				c\in\mathds{R},\text{ } cp(x)=ca_0+ca_1x+ca_2x^2+\dots+ca_nx^n
			\end{matrix}
		\end{equation*}
\end{description}

\chapter {Numeri Complessi}
\newtheorem{NumComp}{Numeri reali}
\begin{NumComp}
	Un numero complesso è definito come un numero della forma $x+iy$, con x e y numeri reali e i una
	soluzione dell'equazione $x^2=-1$ detta unità immaginaria. i numeri reali
	sono
\end{NumComp}
\section{Operazioni con Numeri complessi}
\begin{enumerate}
\item Modulo e distanza
	\begin{equation}
		\abs{z}=\sqrt{x^2+y^2}
	\end{equation}
	Il valore assoluto (modulo) ha proprietà queste proprietà:
	\begin{equation*}
		\abs{z+w}\geq \abs{z}+\abs{w}, \text{ } \abs{zw}=\abs{z}\abs{w}, \text{ } \left|\frac{z}{w}\right|=\frac{\abs z}{\abs w}
	\end{equation*}
	Valide per tutti i numeri complessi $z$ e $w$. La prima proprietà è una versione della disuguaglianza triangolare.

\end{enumerate}

\printindex
\end{document}
