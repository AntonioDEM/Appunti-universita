\documentclass{article}

\usepackage[utf8]{inputenc}
\usepackage{titlesec}
\usepackage{easylist}
\usepackage{hanging}
\usepackage{hyperref}
\usepackage[a4paper,top=2.0cm,bottom=2.0cm,left=2.0cm,right=3.0cm]{geometry}
\usepackage{blindtext}
\usepackage{tipa}
\usepackage{epigraph}
\usepackage{enumerate}
\usepackage{longtable}
\usepackage{setspace}
\usepackage{verbatim}
\usepackage[T1]{fontenc}
\usepackage{graphicx}
\usepackage[italian]{babel}
\usepackage{amsmath}
\usepackage{pbox}
\usepackage{fancyhdr}
\usepackage{cancel}
\usepackage{tabularx}
\usepackage{booktabs}
\usepackage{multirow}
\usepackage{longtable}
\usepackage{tikz}
\usepackage{tikz-qtree}
\usepackage{subfig}
\usepackage{xcolor}
\usepackage{amssymb}
\usepackage{amsmath}
\usepackage{mathrsfs}
\usepackage{textcomp}
\usepackage{circuitikz}
\usepackage{pifont}
\usepackage{imakeidx}
\usepackage{verbatim}
\usepackage{dsfont}
\usepackage{listings}
\usepackage{color}
\usepackage{upgreek}
\usepackage{tasks}
\usepackage{exsheets}
\SetupExSheets[question]{type=exam}

\definecolor{mygreen}{rgb}{0,0.6,0}
\definecolor{mygray}{rgb}{0.5,0.5,0.5}
\definecolor{mymauve}{rgb}{0.58,0,0.82}

\lstset{ 
  backgroundcolor=\color{white},   % choose the background color; you must add \usepackage{color} or \usepackage{xcolor}; should come as last argument
  basicstyle=\footnotesize,        % the size of the fonts that are used for the code
  breakatwhitespace=false,         % sets if automatic breaks should only happen at whitespace
  breaklines=true,                 % sets automatic line breaking
  captionpos=b,                    % sets the caption-position to bottom
  commentstyle=\color{mygreen},    % comment style
  deletekeywords={...},            % if you want to delete keywords from the given language
  escapeinside={\%*}{*)},          % if you want to add LaTeX within your code
  extendedchars=true,              % lets you use non-ASCII characters; for 8-bits encodings only, does not work with UTF-8
  firstnumber=1000,                % start line enumeration with line 1000
  frame=single,	                   % adds a frame around the code
  keepspaces=true,                 % keeps spaces in text, useful for keeping indentation of code (possibly needs columns=flexible)
  keywordstyle=\color{blue},       % keyword style
  language=Octave,                 % the language of the code
  morekeywords={*,...},            % if you want to add more keywords to the set
  numbers=left,                    % where to put the line-numbers; possible values are (none, left, right)
  numbersep=5pt,                   % how far the line-numbers are from the code
  numberstyle=\tiny\color{mygray}, % the style that is used for the line-numbers
  rulecolor=\color{black},         % if not set, the frame-color may be changed on line-breaks within not-black text (e.g. comments (green here))
  showspaces=false,                % show spaces everywhere adding particular underscores; it overrides 'showstringspaces'
  showstringspaces=false,          % underline spaces within strings only
  showtabs=false,                  % show tabs within strings adding particular underscores
  stepnumber=2,                    % the step between two line-numbers. If it's 1, each line will be numbered
  stringstyle=\color{mymauve},     % string literal style
  tabsize=2,	                   % sets default tabsize to 2 spaces
  title=\lstname                   % show the filename of files included with \lstinputlisting; also try caption instead of title
}

\linespread{1.2} % l'interlinea

\frenchspacing

\newcommand{\abs}[1]{\lvert#1\rvert}

\usepackage{floatflt,epsfig}

\usepackage{multicol}
\newcommand\yellowbigsqcup[1][\displaystyle]{%
  \fboxrule0pt
  \ifx#1\textstyle\fboxsep-0.6pt\else\fboxsep-1.25pt\fi
  \mathrel{\fcolorbox{white}{yellow}{$#1\bigsqcup$}}}

\title{Matematica esercizi}
\author{Nicola Ferru}
\date{}
\makeindex[columns=3, title=Alphabetical Index, intoc]

\begin{document}
\maketitle
	\chapter{Testi}
	\begin{equation*}
          \lim_{x\to 0}\frac{x^2+3\sin2x}{x-2\sin3x}
    \end{equation*}
	\begin{equation}
		\lim_{x\to 0} \frac{1-e^{x^2}}{x^3+\sqrt{x}}
	\end{equation}
	\chapter{Soluzioni}
	\begin{equation*}
          \lim_{x\to 0}\frac{x^2+3\sin2x}{x-2\sin3x}=\frac{0^2+3\sin2(0)}{0-2\sin3(0)}
	\end{equation*}
	\begin{equation*}
		\lim_{x\to 0}\frac{1-e^{x^2}}{x^3+\sqrt{x}}=\frac{1-e^0}{0^3+\sqrt{0}}=\frac{0}{0}
	\end{equation*}
	\section{studio di furnzione}
	\begin{equation*}
		f(x)=\frac{e^x}{e^x-1}
	\end{equation*}
	\subsubsection{Dominio}
	\begin{equation*}
		x\neq 0
	\end{equation*}
	Quindi da questa osservazione comprendiamo che la funzione non esiste
	nell'origine.
	\begin{equation*}
		\forall x \in \left(-\infty, 0\right) \vee \left(0, +\infty\right)
	\end {equation*}
	\subsection{simmetria}
		la funzione non è né pari né dispari
	\subsection{intersezione con gli assi}
	\begin{equation*}
		asse x =\begin{cases}
			y=\frac{e^x}{e^x-1}\\
			y=0
		\end{cases}\begin{cases}
			\frac{e^x}{e^x-1}=0\\
			y=0
		\end{cases}
	\end{equation*}
	non interseca nessuno dei due assi
	\subsection{Segno}
		\begin{equation*}
			\frac{e^x}{e^x-1}>0
		\end{equation*}
	$x>0$ perché al denominatore è presente un esponenziale.
	\subsection{Comportamento all'estremo del dominio}
		\begin{equation*}
			\begin{matrix}
				\lim_{x\to
				-\infty}\frac{e^x}{e^x-1}=\frac{e^{-\infty}}{e^{-\infty}-1}=\frac{0}{-1}=0\\
				\lim_{x\to 0^-} \frac{e^x}{e^x-1}=\frac{e^0}{e^0-1}=-\infty\\
				\lim_{x\to 0^+} \frac{e^x}{e^x-1}=\frac{e^0}{e^0-1}=+\infty\\
				\lim_{x\to
				+\infty}\frac{e^x}{e^x-1}=\frac{e^{+\infty}}{e^{+\infty}-1}=\infty
			\end{matrix}
		\end{equation*}
	\subsection{Derivata prima}
		\begin{equation*}
			F^\prime=\frac{e^x*(e^x-1)-e^x*(e^x)}{(e^x)^2}=\frac{-e^x}{(e^x-)^2}
		\end{equation*}
	\subsection{es.2}
		\begin{equation*}
			f(x)=\frac{\ln(2x)}{x}
		\end{equation*}
		\begin{enumerate}
			\item Dominio
				\begin{equation*}
					\begin{matrix}
						x>0\\
						\forall x\in (0; +\infty)
					\end{matrix}
				\end{equation*}
			\item Parità
				\begin{equation*}
					\begin{matrix}
						\neq f(-x) \text{ pari}\\
						\neq -f(x) \text{ dispari}
					\end{matrix}
				\end{equation*}
			\item intersezioni con gli assi
				\begin{equation*}
					asse y\begin{cases}
						f(x)=\frac{\ln(2x)}{x}\\
						x=0
					\end{cases}
				\end{equation*}
				Non interseca l'asse delle ordinate $f(0)=\nexists$
			\item segno
				\begin{equation}
					f(x)=\frac{\ln(2x)}{x}\begin{cases}
						N \leq 0 \to x&\to \frac{1}{2}\\
						D > 0 &\to x>0
					\end{cases}
				\end{equation}
				\begin{equation*}
					f(x)
				\end{equation*}
			\item Comportamento 
		\end{enumerate}
	\subsection{es.4}
	\begin{equation*}
		f(x)=\frac{e^x-2}{x}
	\end{equation*}
	\begin{itemize}
		\item Dominio
		\begin{equation*}
			x\neq 0
		\end{equation*}
		\begin{equation*}
				\forall x \in (-\infty, 0) \vee (0,\infty)
		\end{equation*}
		\item simmetrie 
			\begin{equation*}
				f(x)\begin{cases}
					\neq -f(x)\\
					\neq f(-x)
				\end{cases}
			\end{equation*}
		\item Int. con gli assi
			\begin{equation*}
				asse x\begin{cases}
					y=\frac{e^x-2}{x}\\
					y=0
				\end{cases}\begin{cases}
					\frac{e^2-2}{\not{x}}=0
				\end{cases}\begin{cases}
					e^x=2
				\end{cases}\begin{cases}
					x=\ln 2
				\end{cases}
			\end{equation*}
			asse y\begin{equation*}
				\begin{cases}
					y=\frac{e^x-2}{x}\\
					x=0
				\end{cases}\begin{cases}
					y=\frac{e^0-2}{0}\\
					x=0
				\end{cases}\begin{cases}
					y=\ln2
				\end{cases}
			\end{equation*}
		\item segno
			\begin{equation*}
				\begin{matrix}
					\frac{e^x-2}{x}>0\\
					x>\ln 2
				\end{matrix}
			\end{equation*}
		\item comportamento agli estremi
		\begin{eqnarray}
			\lim_{x\to -\infty}\frac{e^x-2}{x}=\frac{0-2}{-\infty}=0\\
			\lim_{x\to 0}\frac{e^x-2}{x}=\frac{e^0-2}{0}=\infty\\
			\lim_{x\to \infty}\frac{e^x-2}{x}=\frac{\infty-2}{\infty}=\frac{e^x}{1}
		\end{eqnarray}
	\end{itemize}
	\section{es.3}
		\begin{equation*}
			f(x)=\frac{x-1}{x^2-x-6}
		\end{equation*}
		\begin{enumerate}
			\item dominio $\forall x\in \mathds{R}\textbackslash{}\left\{-2,3\right\}$
			\item simmetrie $f(x)\begin{cases}
				\neq -f(x)\\
				\neq f(-x)
			\end{cases}$
			\item Int. con gli assi
				\begin{equation*}
					asse x\begin{cases}
						y=\frac{x-1}{x^2-x-6}\\
						y=0
					\end{cases}\begin{cases}
						\frac{x-1}{x^2-x-6}=0\\
						y=0
					\end{cases}\begin{cases}
						x=1\\
						y=0
					\end{cases}
				\end{equation*}
				\begin{equation*}
					asse y\begin{cases}
						y=\frac{x-1}{x^2-x-6}\\
						x=0
					\end{cases}\begin{cases}
						y=\frac{0-1}{0^2-0-6}\\
						x=0
					\end{cases}\begin{cases}
						y=\frac{1}{6}\\
						x=0
					\end{cases}
				\end{equation*}
			\item segni $-2<x<1\vee x>3$
			\item comportamento agli estremi
				\begin{equation}
					\begin{matrix}
						\lim_{x\to
						-\infty}\frac{x-1}{x^2-x-6}=\frac{-\infty-1}{+\infty}=\frac{\infty}{\infty}
						\to \lim_{x\to\infty}\frac{1}{2x-1}=\frac{1}{\infty}=0 \text{ Assintoto orizzontale}
						\lim_{x\to -2}\frac{x-1}{x^2-x-6}=\frac{-3}{0}=\infty \text{ Assintoto verticale}\\
						\lim_{x\to 3}
						\frac{x-1}{x^2-x-6}=\frac{2}{9-3-6}=\infty \text{ Assintoto verticale}
					\end{matrix}
				\end{equation}
		\end{enumerate}


	\section{teorema di Roll}
		\begin{equation*}
			f(x)=x^2-4x+3
		\end{equation*}
	\begin{equation*}
		y=[-1,5]
	\end{equation*}
		\begin{equation*}
			(-1)^2-4(-1)+3=1+4+3=8
		\end{equation*}
		\begin{equation*}
			(5)^2-4(5)+3=25-20+3=8
		\end{equation*}
		\begin{equation}
	.		\begin{matrix}
				f(c)=0\\
				f^\prime(x)=2x-4\\
				2c-4=0\\
				\frac{\not{2}c}{\not{2}}=\frac{\not4}{\not2}\to c=2
			\end{matrix}
		\end{equation}
		\subsection{es.2}
		\begin{equation*}
			f(x)=x^4+x^2+1
		\end{equation*}
		Intervallo compreso tra $[-2,2]$
		\begin{equation*}
			\begin{matrix}
				f(-2)=(-2)^4+(-2)^2+1=16+4+1=21\\
				f(2)=(2)^4+(2)^2+1=16+4+1=21
			\end{matrix}
		\end{equation*}
		la funzione è continua
		\begin{equation*}
			\begin{matrix}
				f^\prime(x)=4x^3+2x\\
				f^\prime(c)=0\\
				x=c\\
				4c^3-2c=0 \to 2c(2c^2+1)=0\\
				2c=0\to c=0\\
				2x^2+1=0\to c=\pm\sqrt{-\frac{1}{2}} \text{ NO}
			\end{matrix}
		\end{equation*}
	\section{Teorema di Lagrange}
	\begin{equation*}
		f(x)=2x^2+x+1, \text{ } [-2;3]
	\end{equation*}
	\begin{equation*}
		\begin{matrix}
			2(-2)^2-2+1=8-1=7\\
			2(3)^2+3+1=23 \text{ NO}
		\end{matrix}
	\end{equation*}
questa funzione non rispetta i punti del teorema di Lagrange.
\subsection{es.2}
\begin{equation*}
	f(x)=\sqrt{x}-x,\text{ [0,4]}
\end{equation*}
\begin{equation*}
	\begin{matrix}
		\sqrt{0}-0=0\\
		\sqrt{4}-4=2
	\end{matrix}
\end{equation*}
la funzione è continua
\begin{equation*}
	f^\prime=\frac{1}{2\sqrt{x}}-1
\end{equation*}
la funzione è derivabile
\begin{equation*}
	f^\prime(c)=\frac{f(b)-f(a)}{b-a}=\frac{-2-0}{4-0}=-\frac{1}{2}
\end{equation*}
\begin{equation*}
	\boxed{\frac{1}{2\sqrt{x}}=-\frac{1}{2}}
\end{equation*}
\section{Equazione differenziali}
\begin{equation*}
	y^\prime+2xy=x\sin(x^2)
\end{equation*}
\begin{equation*}
	x^\prime=-2xy+x\sin(x^2)
\end{equation*}
\begin{equation*}
	y^\prime=a(x)b(x)
\end{equation*}
\begin{equation*}
	y^\prime=-2x+x\sin(x^2)
\end{equation*}
\begin{equation*}
	y(x)=e^{-A(x)}\left(c+\int e^{A(x)}f(x)dx\right)
\end{equation*}
\begin{equation*}
	\begin{matrix}
		a(x)=2x;&A(x)=\int a(x)dx=x^2
	\end{matrix}
\end{equation*}
\begin{equation}
	y=e^{-x^2}\left\{c+\int e^{x^2}x\sin x^2\right\}
\end{equation}
\begin{equation}
	\int e^{x^2}x\sin x^2dx=\left[x^2=t;dx=\frac{dt}{2}\right]=\frac{1}{2}\int e^x\sin tdt
\end{equation}
(con integrazione per parti standard)
\begin{equation}
	=\frac{1}{2}e^t(\sin t-\cos t)=\frac{1}{4}e^{x^2}(\sin(x^2)-\cos(x^2))
\end{equation}
\begin{equation*}
	y=e^{e^t}\left\{c+\frac{1}{4}e^{x^2}(\sin x^2-\cos x^2)\right\}=ce^{-x^2}+\frac{1}{4}(\sin x^2-\cos x^2)
\end{equation*}
\section{integrali di secondo tipo}
\begin{equation*}
	y^{\prime\prime}-y^\prime+2y=3xe^{-x}
\end{equation*}
\begin{equation*}
	\begin{matrix}
		t=y^\prime \\
		t^2-3t+2=3xe^{-x} \\
		\Delta=b^2-4(a)(c)=9-4(1)(2)=1 \\
		t_{1,2}=\frac{-b\pm\sqrt{\Delta}}{2a}=\frac{3\pm{1}}{2}=\begin{cases}
			\frac{3+{1}}{2}=2 \\
			\frac{3-{1}}{2}=1
		\end{cases}\\
		z(x)=c_1e^x+c_2e^{2x}\\
		f(x)=3xe^{-x}\\
		y(x)=(ax+b)e^{-x}\\
		y^\prime=e^{-x}(-ax-b+a)\\
		y^{\prime\prime}=e^{-x}(ax+b-2a)\\
		e^{-x}[6ax+(6b-5a)]=3xe^{-x}\\
		\begin{cases}
			6a=3\\
			6b-5a=0
		\end{cases}&a=\frac{1}{2};&b=\frac{5}{12};\\
		y(x)=\left(\frac{1}{2}x+\frac{5}{12}\right)e^{-x}\\
		c_1e^x+c_2e^{2x}+\left(\frac{1}{2}x+\frac{5}{12}\right)e^{-x}
	\end{matrix}
\end{equation*}
\section{integrali}
\begin{equation*}
	\begin{matrix}
		\int\frac{3x-4}{x^2-6x+8}dx\\
		\Delta=b^2-4(a)(c)=36-32=4\\
		\int\frac{R(x)}{D(x)}=\frac{A}{x-x^1}dx+\int\frac{B}{x-x_2}dx=A\ln(\abs{x-x_2})+B(\abs{x+1})\\
		\int\frac{3x-4}{x^2-6x+8}dx=\frac{3x-4}{(x-4)(x-2)}=\frac{A}{x-4}+\frac{B}{x-2}=\frac{A(x-2)+B(x-4)}{(x-4)(x-2)}\\
		A(x-2)+B(x-4)-2A-4B\Leftrightarrow\begin{cases}
			(A+B)x=3 \\
			-2A-4B=-4
		\end{cases}\Rightarrow\begin{cases}
			A=-B+3\\
			-2(-B+3)-4B=-4
		\end{cases}\\\Rightarrow\begin{cases}
			A=-B+3\\
			2B-6-4B=-4
		\end{cases}\Rightarrow\begin{cases}
			A=-B+3\\
			-2B=2
		\end{cases}\Rightarrow\begin{cases}
			A=-B+3\\
			\frac{-2B}{2}=\frac{2}{2}
		\end{cases}\Rightarrow\begin{cases}
			A=-B+3\\
			B=-1
		\end{cases}\\
		\Rightarrow\begin{cases}
			A=1+3\to 4\\
			B=-1
		\end{cases}\\
		\frac{3x-4}{x^2-6x+8}dx=\frac{4}{x-4}-\frac{1}{x-2}=\int\left[\frac{4}{x-4}\right]dx=4\log\abs{x-4}-\log\abs{x-2}+c
	\end{matrix}
\end{equation*}


\end{document}
