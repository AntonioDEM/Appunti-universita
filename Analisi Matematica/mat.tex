\documentclass{book}

\usepackage[utf8]{inputenc}
\usepackage{titlesec}
\usepackage{easylist}
\usepackage{hanging}
\usepackage{hyperref}
\usepackage[a4paper,top=2.0cm,bottom=2.0cm,left=2.0cm,right=3.0cm]{geometry}
\usepackage{blindtext}
\usepackage{tipa}
\usepackage{epigraph}
\usepackage{enumerate}
\usepackage{longtable}
\usepackage{setspace}
\usepackage{verbatim}
\usepackage[T1]{fontenc}
\usepackage{graphicx}
\usepackage[italian]{babel}
\usepackage{amsmath}
\usepackage{pbox}
\usepackage{fancyhdr}
\usepackage{cancel}
\usepackage{tabularx}
\usepackage{booktabs}
\usepackage{multirow}
\usepackage{longtable}
\usepackage{tikz}
\usepackage{tikz-qtree}
\usepackage{subfig}
\usepackage{xcolor}
\usepackage{amssymb}
\usepackage{mathrsfs}
\usepackage{textcomp}
\usepackage{circuitikz}
\usepackage{pifont}
\usepackage{imakeidx}
\usepackage{verbatim}

\usepackage{listings}
\usepackage{color}

\definecolor{mygreen}{rgb}{0,0.6,0}
\definecolor{mygray}{rgb}{0.5,0.5,0.5}
\definecolor{mymauve}{rgb}{0.58,0,0.82}

\lstset{ 
  backgroundcolor=\color{white},   % choose the background color; you must add \usepackage{color} or \usepackage{xcolor}; should come as last argument
  basicstyle=\footnotesize,        % the size of the fonts that are used for the code
  breakatwhitespace=false,         % sets if automatic breaks should only happen at whitespace
  breaklines=true,                 % sets automatic line breaking
  captionpos=b,                    % sets the caption-position to bottom
  commentstyle=\color{mygreen},    % comment style
  deletekeywords={...},            % if you want to delete keywords from the given language
  escapeinside={\%*}{*)},          % if you want to add LaTeX within your code
  extendedchars=true,              % lets you use non-ASCII characters; for 8-bits encodings only, does not work with UTF-8
  firstnumber=1000,                % start line enumeration with line 1000
  frame=single,	                   % adds a frame around the code
  keepspaces=true,                 % keeps spaces in text, useful for keeping indentation of code (possibly needs columns=flexible)
  keywordstyle=\color{blue},       % keyword style
  language=Octave,                 % the language of the code
  morekeywords={*,...},            % if you want to add more keywords to the set
  numbers=left,                    % where to put the line-numbers; possible values are (none, left, right)
  numbersep=5pt,                   % how far the line-numbers are from the code
  numberstyle=\tiny\color{mygray}, % the style that is used for the line-numbers
  rulecolor=\color{black},         % if not set, the frame-color may be changed on line-breaks within not-black text (e.g. comments (green here))
  showspaces=false,                % show spaces everywhere adding particular underscores; it overrides 'showstringspaces'
  showstringspaces=false,          % underline spaces within strings only
  showtabs=false,                  % show tabs within strings adding particular underscores
  stepnumber=2,                    % the step between two line-numbers. If it's 1, each line will be numbered
  stringstyle=\color{mymauve},     % string literal style
  tabsize=2,	                   % sets default tabsize to 2 spaces
  title=\lstname                   % show the filename of files included with \lstinputlisting; also try caption instead of title
}

\linespread{1.2} % l'interlinea

\frenchspacing

\newcommand{\abs}[1]{\lvert#1\rvert}

\usepackage{floatflt,epsfig}

\usepackage{multicol}
\newcommand\yellowbigsqcup[1][\displaystyle]{%
  \fboxrule0pt
  \ifx#1\textstyle\fboxsep-0.6pt\else\fboxsep-1.25pt\fi
  \mathrel{\fcolorbox{white}{yellow}{$#1\bigsqcup$}}}

\title{Appunti di Matematica}
\author{Nicola Ferru}
\date{}
\makeindex[columns=3, title=Alphabetical Index, intoc]

\begin{document}
\maketitle
\tableofcontents
\part{Matematica analisi 1}
\section{Simboli}
\begin{multicols}{3}
 $\in$ Appartiene\\
 $\notin$ Non appartiene\\
 $\exists$ Esiste\\
 $\exists !$ Esiste unico\\
 $\subset$ Contenuto strettamente\\
 $\subseteq$ Contenuto\\
 $\supset$ Contenuto strettamente\\
 $\supseteq$ Contiene\\
$\Rightarrow$ Implica\\
$\Longleftrightarrow$ Se e solo se\\
$\neq$ Diverso\\
$\forall$ Per ogni\\
$\ni :$ Tale che\\
$\leq$ Minore o uguale\\
$\geq$ Maggiore o uguale\\
$\alpha$ alfa\\
$\beta$ beta\\
$\gamma$ gamma\\
$\Gamma$ Gamma\\
$\delta,\Delta$ delta\\
$\epsilon$ epsilon\\
$\sigma,\Sigma$ sigma\\
$\rho$ rho
\end{multicols}
\chapter{Studio di funzione}
\section{Cenni di teoria degli insiemi}
Per rappresentare un insieme abbiamo tre possibilità:
\begin{enumerate}
	\item Rappresentazione estensive $A=[0,1,2,3,4]$
	\item Rappresentazione intensiva $A=[x|x\in N e x<5]$
	\item Rappresentazione con diagrammi di Eulero - Venn\\
	\begin{tikzpicture}domain=0:10] 
		 \draw (0,0) ellipse (2cm and 1cm);
    		 \filldraw  (0,0.3) circle (2pt) node[align=left,   below] {1}
    			(1,0.11) circle (2pt) node[align=left,   below] {2}
    			(-1,-0.4) circle (2pt) node[align=left,   below] {0}
    			(-1,0.1) circle (2pt) node[align=left,   below] {3}
    			(0.1,-0.5) circle (2pt) node[align=left,   below] {4};
		\end{tikzpicture}
\end{enumerate}
\subsection{Operazioni tra gli insiemi}
Un insieme può essere contenuto in un altro:\\
\begin{tikzpicture}domain=0:10] 
    \draw (0,0) ellipse (2cm and 1cm);
     \draw (0,0) ellipse (0.7cm and 0.5cm);

    \filldraw  (0,0.3) circle (2pt) node[align=left,   below] {1}
    (-0.52,0.3) node[align=left,   below] {B}
    (0.55,0.11) circle (2pt) node[align=left,   below] {2}
    (-1,-0.4) circle (2pt) node[align=left,   below] {0}
    (-1,0.1) circle (2pt) node[align=left,   below] {3}
    (0.6,-0.55) circle (2pt) node[align=left,   below] {4};
\end{tikzpicture}

\section{Limiti}
\subsection{Infinitesimi e infiniti}
\paragraph{Definizione}
Una funzione $f(x)$ su dice \underline{infinitesima} per $x\to x_0$ (per $x\to \infty$), $x_0$ punto di accumulazione per il dominio di $f(x)$, se: $\lim_{x\to x_0}f(x)=0$ (oppure $lim_{x\to \infty}f(x)=0$).
\subsubsection{Esempi}
\begin{itemize}
	\item $y=e^x$ è un infinitesimo per $x\to -\infty$
	\item $y=\ln{x}$ è un infinitesimo per $x\to 1$
	\item $y=\sin{x}$ è un infinitesimo per $x\to 0$ (ma anche per $x\to \pi,2\pi$, etc.)
	\item $y=\ln{1+x}$ è un infinitesimo per $x\to 1$ 
\end{itemize}
\subsubsection{Ordine di infinitesimo}
Siano $f(x)$ e $g(x)$ infinitesimi per $x\to{x_0}$ (o per $x\to \infty$), con $g(x)\neq 0$. Se $\exists\alpha R+$ e $l\in R$, $l\neq 0$ tale che\\
$\lim_{x\to{x_0}}=\frac{f(x)}{[g(x)]^\alpha}=l$ (oppure $\lim_{x\to{\infty}}=\frac{f(x)}{[g(x)]^\alpha}=l$)\\
Allora, si dice che per $x\to x_0$ (o per $x\to \infty$), $f(x)$ è un infinitesimo di ordine $\alpha$ rispetto all'infinitesimo campione $g(x)$.
\paragraph{Esempi}
\begin{itemize}
	\item $y=\sin{x}$ è un infinitesimo per $x\to 0$ di ordine 1 rispetto all'infinitesimo campione $g(x)=x$, infatti, $\lim_{x\to 0}=\frac{\sin{x}}{x^\alpha}=1$ solo se $\alpha = 1$
	\item $y=\tan^2x$ è un infinitesimo di ordine 2 rispetto ad x, per $x\to 0$
	\item $ord(l-\cos{x})=2$ rispetto ad $x$ per $x\to 0$
\end{itemize}
\subsubsection{Confronto tra infinitesimi}
Siano $f(x)$ e $g(x)$ infinitesime per $x\to x_{0}$,\\
$\lim_{x\to x_0}\frac{f(x)}{g(x)}=
\begin{cases}
l\neq 0&ord(f)=ord(g)\\
\pm \infty&ord(f)<ord(g)\\
0&ord(f)>ord(g)\\
non esiste, & f e g non confrontabile \\ 
\end{cases}
$\\
Stesso risultato se $f(x)$ e $g(x)$ sono infinitesime per $x\to \infty$. Utilizzando il confronto tra infinitesimi nel calcolo dei limiti del tipo $\lim_{x\to x_0}\frac{f_1+f_2}{g_1+g_2}$, dove $f_1,f_2,g_1,g_2$ sono funzioni infinitesime per $x\to x_0$, si possono {\color{blue} \em trascurare gli infinitesimi di ordine maggiore} (analogo discorso per funzioni infinitesime $x\to \infty$).
\subparagraph{esempio}
$\lim_{x\to 0}\frac{x^2+x^3+2\tan{x}}{(e^x-1)^2+\sin{x}}=\lim_{x\to 0}\frac{2\tan x}{\sin x}=2$
\paragraph{Definizione di funzioni asintotiche}
Si dice che due funzioni $f,g$ sono asintotiche per $x\to x_0$ se $\lim_{x\to x_0}\frac{f(x)}{g(x)}=1$ e si scrive $f\sim g$ per $x\to x_0$
\subparagraph{esempi}
\begin{itemize}
	\item $\sin x\sim x$ per $x\to 0$
	\item $\ln(1+x)\sim x$ per $x\to 0$
	\item $e^x-1\sim x$ per $x\to 0$ 
\end{itemize}
\paragraph{Definizione di funzioni infinite}
Una funzione $f(x)$ si dice \textit{infinita} per $x\to x_0$ (o per $x \to \infty$) , $x_0$ punto di accumulazione per il dominio di $f(x)$, (o per $x\to \infty$) se:
\begin {center}
	$\lim_{x\to x_0}f(x)=\infty$ (oppure $\lim_{x\to \infty}f(x)=\infty$)
\end{center}
\subparagraph{Esempi}
\begin{itemize}
	\item $y=e^x$ è un infinito per $x\to +\infty$
	\item $y=\ln{x}$ è un infinito per $x\to 0^+$
	\item $y=x^2+x$ è un infinito per $x\to \infty$
\end{itemize}
\subparagraph{Regole aritmetiche}
Siano $f(x)=o(x^\alpha)$ (si legge <<o piccolo di>>) e $g(x)=o(x^\beta)$ due funzioni \textit{infinitesime} rispettivamente di ordine $\alpha$ e $\beta$ per $x \to 0$ Allora si ha
\begin{itemize}
	\item $cf(x))o(x^\alpha)$, $\forall c \in R$
	\item $x^\lambda f(x)=o(x^{\lambda+\alpha})$
	\item $f(x)g(x)=o(x^{\alpha+\beta})$
	\item $f(x)+g(x)=o(x^y)$, $\gamma=min(\alpha,\beta)$
\end{itemize}
\paragraph{Ordine di infinito}
Siamo $f(x)$ e $g(x)$ infiniti per $x\to x_0$ (o per $x$), con $g\neq 0$. Se $\exists\alpha\in R+$ e $l\in R$, $l\neq 0$ tale che\\
$\lim_{x\to x_0}\frac{f(x)}{[g(x)]^2}=l$ (o $\lim_{x\to \infty}\frac{f(x)}{[g(x)]^\alpha}=l$)\\
Allora, per $x\to x_0$ (o per $x\to \infty$), $f(x)$ è un infinito di ordine $\alpha$ rispetto all'infinito compione $g(x)$.
\subparagraph{Esempi}
\begin{itemize}
	\item $ord(\sqrt{x})=\frac{1}{2}$ rispetto ad $x$ per $x\to +\infty$
	\item $ord(\frac{1}{\sin x})=1$ rispetto ad $\frac{1}{x}$ per $x\to 0$
	\item $ord(\frac{1}{e^x-1})=1$ rispetto ad $\frac{1}{x}$ per $x\to 0$
\end{itemize}
\subparagraph{Cofronto tra infiniti}
Siamo $f(x)$ e $g(x)$ infiniti per $x\to x_0$\\
$\lim_{x\to x_0}\frac{f(x)}{g(x)}=\begin{cases}
l\neq 0&ord(f)=ord(g)\\
\pm \infty&ord(f)>ord(g)\\
0&ord(f)<ord(g)\\
non esiste, & f e g non confrontabile \\ 
\end{cases}
$\\
Stesso risultato se $f(x)$ e $g(x)$ sono infinite per $x \to \infty$. Utilizzando il confronto tra infiniti nel calcolo dei limiti del tipo $\lim_{x\to x_0}\frac{f_1+f_2}{g_1+g_2}$, deve $f_1,f_2,g_1,g_2$ sono funzioni infinite per $x\to x_0$, si possono {\color{red}trascurare gli \underline{infiniti} di ordine minore} (analogo discorso per funzione infinito $x\to \infty$).
\subparagraph{Esempio}
$\lim_{x\to +\infty}\frac{x^2+x^3+3\sqrt{x}}{x^2(2x-1)+\sqrt{3x}}=\lim_{x\to +\infty}\frac{x^3}{2x^3}=\frac{1}{2}$.
\paragraph{Gerarchia degli infiniti}
Per $x\to +\infty$ si ha $(\log_\alpha x)^\alpha<<x^\beta<<b^x$, con $\alpha,\beta>0,a,b>1$ Non sempre è possibile calcolare l'ordine di infinito (o di infinitesimo) rispetto alla funzione campione usuale.\\
\subparagraph{Esempio}
$\lim_{x\to +\infty}\frac{a^x}{x^a}=+\infty, \forall \alpha>0, a>1$, $\lim_{x\to +\infty}\frac{(\log_a x)^\beta}{x^a}=+\infty, \forall \alpha, \beta>0, a>1$
\subparagraph{Regole aritmetiche}
Siano $f(x)$ e $g(x)$ due funzioni \emph{infinite} di ordine rispettivamente $\alpha$ e $\beta$. Allora si ha
\begin{itemize}
	\item $ord(f(x)+g(x))=\max{\alpha,\beta}$
	\item $ord(f(x)*g(x))=\alpha+\beta$
	\item $ord((f(x))^\gamma)=\alpha\gamma$
\end{itemize}
\printindex
\end{document}