\documentclass{article}

\usepackage[utf8]{inputenc}
\usepackage{titlesec}
\usepackage{easylist}
\usepackage{hanging}
\usepackage{hyperref}
\usepackage[a4paper,top=2.0cm,bottom=2.0cm,left=2.0cm,right=2.0cm]{geometry}
\usepackage{blindtext}
\usepackage{tipa}
\usepackage{epigraph}
\usepackage{enumerate}
\usepackage{longtable}
\usepackage{setspace}
\usepackage{verbatim}
\usepackage[T1]{fontenc}
\usepackage{graphicx}
\usepackage[italian]{babel}
\usepackage{amsmath}
\usepackage{pbox}
\usepackage{fancyhdr}
\usepackage{cancel}
\usepackage{tabularx}
\usepackage{booktabs}
\usepackage{multirow}
\usepackage{longtable}
\usepackage{tikz}
\usepackage{tikz-qtree}
\usepackage{subfig}
\usepackage{xcolor}
\usepackage{amssymb}
\usepackage{mathrsfs}
\usepackage{textcomp}

\usepackage{listings}
\usepackage{color}

\definecolor{mygreen}{rgb}{0,0.6,0}
\definecolor{mygray}{rgb}{0.5,0.5,0.5}
\definecolor{mymauve}{rgb}{0.58,0,0.82}

\lstset{ 
  backgroundcolor=\color{white},   % choose the background color; you must add \usepackage{color} or \usepackage{xcolor}; should come as last argument
  basicstyle=\footnotesize,        % the size of the fonts that are used for the code
  breakatwhitespace=false,         % sets if automatic breaks should only happen at whitespace
  breaklines=true,                 % sets automatic line breaking
  captionpos=b,                    % sets the caption-position to bottom
  commentstyle=\color{mygreen},    % comment style
  deletekeywords={...},            % if you want to delete keywords from the given language
  escapeinside={\%*}{*)},          % if you want to add LaTeX within your code
  extendedchars=true,              % lets you use non-ASCII characters; for 8-bits encodings only, does not work with UTF-8
  firstnumber=1000,                % start line enumeration with line 1000
  frame=single,	                   % adds a frame around the code
  keepspaces=true,                 % keeps spaces in text, useful for keeping indentation of code (possibly needs columns=flexible)
  keywordstyle=\color{blue},       % keyword style
  language=Octave,                 % the language of the code
  morekeywords={*,...},            % if you want to add more keywords to the set
  numbers=left,                    % where to put the line-numbers; possible values are (none, left, right)
  numbersep=5pt,                   % how far the line-numbers are from the code
  numberstyle=\tiny\color{mygray}, % the style that is used for the line-numbers
  rulecolor=\color{black},         % if not set, the frame-color may be changed on line-breaks within not-black text (e.g. comments (green here))
  showspaces=false,                % show spaces everywhere adding particular underscores; it overrides 'showstringspaces'
  showstringspaces=false,          % underline spaces within strings only
  showtabs=false,                  % show tabs within strings adding particular underscores
  stepnumber=2,                    % the step between two line-numbers. If it's 1, each line will be numbered
  stringstyle=\color{mymauve},     % string literal style
  tabsize=2,	                   % sets default tabsize to 2 spaces
  title=\lstname                   % show the filename of files included with \lstinputlisting; also try caption instead of title
}

\linespread{1.5} % l'interlinea

\frenchspacing

\newcommand{\abs}[1]{\lvert#1\rvert}

\usepackage{floatflt,epsfig}

\usepackage{multicol}
\newcommand\yellowbigsqcup[1][\displaystyle]{%
  \fboxrule0pt
  \ifx#1\textstyle\fboxsep-0.6pt\else\fboxsep-1.25pt\fi
  \mathrel{\fcolorbox{white}{yellow}{$#1\bigsqcup$}}}

\title{Formulario}
\author{Nicola Ferru}
\begin{document}
\maketitle
\section{Derivate}
\begin{multicols}{2}
	\begin{itemize}
		\item $D(x^n)=n*x^{n-1}$
		\item $D(\log_ax)=\frac{1}{x}\log_a e$
		\item $D(a^x)=a^x\ln a$
		\item $D(\sin x)=\cos x$
		\item $D(\cos x)=-\sin x$
		\item $D(k)=0$
		\item $D(\ln x)=\frac{1}{x}$
		\item $D(e^x)=e^x$
		\item $D(\tan x)=\frac{1}{\cos^2 x}=1+\tan^2x$
		\item $D(\arcsin x)=\frac{1}{\sqrt{1-x^2}}$
		\item $D(\arccos x)=-\frac{1}{\sqrt{1-x^2}}$
		\item $D(\arctan x)=\frac{1}{1+x^2}$
	\end{itemize}
\end{multicols}
Casi:
\begin{multicols}{2}
	\begin{equation}
		D[k*f(x)]=k*f^\prime(x)
	\end{equation}
	\begin{equation}
		D[f(x)+g(x)+h(x)]=f^\prime+g^\prime+h^\prime
	\end{equation}
	\begin{equation}
		D\left[\frac{f(x)}{g(x)}\right]=\frac{f^\prime*g-f*g^\prime}{\left[ g\right]^2}
	\end{equation}
	\begin{equation}
		D\left[\frac{1}{f(x)}\right]=\frac{f^\prime(x)}{\left[f(x)\right]^2}
	\end{equation}
	\begin{equation}
		D[f(g(x))]=	f^\prime[g(x)]*g^\prime
	\end{equation}
\end{multicols}
\subsection{Limiti Notevoli}
\subsubsection{esponenziali e logaritmici}
\begin{multicols}{2}
\begin{equation}
	\lim_{x\to\pm\infty}\left(1+\frac{1}{x}\right)^x=e
\end{equation}
\begin{equation}
	\lim_{x\to+\infty}\left(1+\frac{a}{x}\right)^x=e^a
\end{equation}
\begin{equation}
	\lim_{x\to+\infty}\left(1+\frac{a}{x}\right)^{nx}=e^{na}
\end{equation}
\begin{equation}
	\lim_{x\to-\infty}\left(1+\frac{a}{x}\right)^x=\frac{1}{e}
\end{equation}
\begin{equation}
	\lim_{x\to0}\left(1+ax\right)^{\frac{1}{x}}=e^{a}
\end{equation}
\begin{equation}
	\lim_{x\to0}\lg_a\left(1+x\right)^\frac{1}{x}=\frac{1}{\lg_e a}
\end{equation}
\begin{equation}
	\lim_{x\to0}\frac{\lg_a\left(1+x\right)}{x}=\lg_ae=\frac{1}{\ln a}
\end{equation}
\begin{equation}
	\lim_{x\to0}\frac{a^x-1}{x}=\ln a
\end{equation}
\begin{equation}
	\lim_{x\to0}\frac{\left(1+x\right)^a-1}{x}=a
\end{equation}
\begin{equation}
	\lim_{x\to0}\frac{\left(1+x\right)^a-1}{x}=1
\end{equation}
\begin{equation}
	\begin{matrix}
		\lim_{x\to0}x^r\lg_a x=0&\forall \in R^+-\{1\},&\forall r\in R^+
	\end{matrix}
\end{equation}
\begin{equation}
	\begin{matrix}
		\lim_{x\to0}\frac{\lg_a x}{x^r}=0&\forall \in R^+-\{1\},&\forall r\in R^+
	\end{matrix}
\end{equation}
\begin{equation}
	\lim_{x\to+\infty}x^ra^x=\lim_{x\to+\infty}a^x
\end{equation}
\begin{equation}
	\lim_{x\to-\infty}\abs{x}^ra^x=\lim_{x\to\infty}a^x
\end{equation}
\begin{equation}
	\begin{matrix}
		\lim_{x\to+\infty}\frac{e^x}{x^r}=\lim_{x\to+\infty}a^x&\forall r \in R^+
	\end{matrix}
\end{equation}
\begin{equation}
	\begin{matrix}
		\lim_{x\to+\infty}\frac{x^x}{e^r}=\lim_{x\to+\infty}a^x&\forall r \in R^+
	\end{matrix}
\end{equation}
\begin{equation}
	\begin{matrix}
		\lim_{x\to-\infty}e^x*x^r=0&\forall r \in R^+
	\end{matrix}
\end{equation}
\end{multicols}
\subsection{Goniometrici}
\begin{multicols}{2}
\begin{equation}
	\lim_{x\to0}\frac{\sin x}{x}=1
\end{equation}
\begin{equation}
	\lim_{x\to0}\frac{\sin ax}{bx}=\frac{a}{b}
\end{equation}
\begin{equation}
	\lim_{x\to0}\frac{\tan x}{x}=1
\end{equation}
\begin{equation}
	\lim_{x\to0}\frac{1-\cos x}{x}=0
\end{equation}
\begin{equation}
	\lim_{x\to0}\frac{1-\cos x}{x^2}=\frac{1}{2}
\end{equation}
\begin{equation}
	\lim_{x\to0}\frac{\arcsin ax}{bx}=\frac{a}{b}
\end{equation}
\begin{equation}
	\lim_{x\to0}\frac{arctan x}{x}=1
\end{equation}
\end{multicols}
\section{formula retta tangente}
\begin{equation}
(y - y_0) = m·( x - x_0)
\end{equation}
\begin{eqnarray}
	m=f^\prime(x_0)
\end{eqnarray}
\paragraph{formula per il massimo e minimo relativo}
\begin{equation}
	f(x_0)
\end{equation}



\end{document}
