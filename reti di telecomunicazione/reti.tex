\documentclass{book}

\usepackage[utf8]{inputenc}
\usepackage{titlesec}
\usepackage{easylist}
\usepackage{hanging}
\usepackage{hyperref}
\usepackage[a4paper,top=2.0cm,bottom=2.0cm,left=2.0cm,right=3.0cm]{geometry}
\usepackage{blindtext}
\usepackage{tipa}
\usepackage{epigraph}
\usepackage{enumerate}
\usepackage{longtable}
\usepackage{setspace}
\usepackage{verbatim}
\usepackage[T1]{fontenc}
\usepackage{graphicx}
\usepackage[italian]{babel}
\usepackage{amsmath}
\usepackage{pbox}
\usepackage{fancyhdr}
\usepackage{cancel}
\usepackage{tabularx}
\usepackage{booktabs}
\usepackage{multirow}
\usepackage{longtable}
\usepackage{tikz}
\usepackage{tikz-qtree}
\usepackage{subfig}
\usepackage{xcolor}
\usepackage{amssymb}
\usepackage{amsmath}
\usepackage{mathrsfs}
\usepackage{textcomp}
\usepackage{circuitikz}
\usepackage{pifont}
\usepackage{imakeidx}
\usepackage{verbatim}
\usepackage{dsfont}
\usepackage{listings}
\usepackage{color}
\usepackage{upgreek}
\usepackage{tasks}
\usepackage{exsheets}
\usepackage{pgfplots}

\SetupExSheets[question]{type=exam}

\definecolor{mygreen}{rgb}{0,0.6,0}
\definecolor{mygray}{rgb}{0.5,0.5,0.5}
\definecolor{mymauve}{rgb}{0.58,0,0.82}

\lstset{ 
  backgroundcolor=\color{white},   % choose the background color; you must add \usepackage{color} or \usepackage{xcolor}; should come as last argument
  basicstyle=\footnotesize,        % the size of the fonts that are used for the code
  breakatwhitespace=false,         % sets if automatic breaks should only happen at whitespace
  breaklines=true,                 % sets automatic line breaking
  captionpos=b,                    % sets the caption-position to bottom
  commentstyle=\color{mygreen},    % comment style
  deletekeywords={...},            % if you want to delete keywords from the given language
  escapeinside={\%*}{*)},          % if you want to add LaTeX within your code
  extendedchars=true,              % lets you use non-ASCII characters; for 8-bits encodings only, does not work with UTF-8
  firstnumber=1000,                % start line enumeration with line 1000
  frame=single,	                   % adds a frame around the code
  keepspaces=true,                 % keeps spaces in text, useful for keeping indentation of code (possibly needs columns=flexible)
  keywordstyle=\color{blue},       % keyword style
  language=Octave,                 % the language of the code
  morekeywords={*,...},            % if you want to add more keywords to the set
  numbers=left,                    % where to put the line-numbers; possible values are (none, left, right)
  numbersep=5pt,                   % how far the line-numbers are from the code
  numberstyle=\tiny\color{mygray}, % the style that is used for the line-numbers
  rulecolor=\color{black},         % if not set, the frame-color may be changed on line-breaks within not-black text (e.g. comments (green here))
  showspaces=false,                % show spaces everywhere adding particular underscores; it overrides 'showstringspaces'
  showstringspaces=false,          % underline spaces within strings only
  showtabs=false,                  % show tabs within strings adding particular underscores
  stepnumber=2,                    % the step between two line-numbers. If it's 1, each line will be numbered
  stringstyle=\color{mymauve},     % string literal style
  tabsize=2,	                   % sets default tabsize to 2 spaces
  title=\lstname                   % show the filename of files included with \lstinputlisting; also try caption instead of title
}

\linespread{1.2} % l'interlinea

\frenchspacing

\newcommand{\abs}[1]{\lvert#1\rvert}

\usepackage{floatflt,epsfig}

\usepackage{multicol}
\newcommand\yellowbigsqcup[1][\displaystyle]{%
  \fboxrule0pt
  \ifx#1\textstyle\fboxsep-0.6pt\else\fboxsep-1.25pt\fi
  \mathrel{\fcolorbox{white}{yellow}{$#1\bigsqcup$}}}

\title{Appunti di Matematica}
\author{Nicola Ferru}
\date{}
\makeindex[columns=3, title=Alphabetical Index, intoc]

\begin{document}
\maketitle
\tableofcontents
\listoftables
\listoffigures
\chapter{Introduzione}

\chapter{Livelli architetturali bassi}

\section{Gestione degli errori}
Visto che i mezzi fisici possono generare degli errori di trasmissione o
recezione, sono stati inventati dei metodi per riuscire a comprendere se
l'informazione trasmessa sia arrivata a destinazione integra. I due metodi
principali sono:
\begin{enumerate}
	\item Controllo e correzione d'errore;
	\item Recupero d'errore.
\end{enumerate}
\subsection{Rivelazione di errore}
\begin{itemize}
	\item Normalmente si basa sull'aggiunta di ridondanza in trasmissione
		\begin{itemize}
			\item utilizzata in ricezione per rivelare (ma non correggere)
				gli errori;
			\item la ridondanza richiesta per la rivelazione è molto più
				contenuta rispetto a quella che sarebbe richiesta per la
				correzione (16-32bit)
		\end{itemize}
	\item Può essere alla base di un'eventuale correzione/recupero
	\item Differenti meccanismi di gestione del codice di rivelazione di errore
		\begin{itemize}
			\item controllo di parità (a blocchi), somma completo a 1
				(\textit{checksum}), etc.
		\end{itemize}
	\item un codice di rivelazione di errore deve rilevare solo modifiche
		casuali.
\end{itemize}
\subsection{Controllo di parità}
\begin{itemize}
	\item Per ogni blocco di bit viene aggiunto un bit pari se il numero di 1
		nel blocco è dispari, altrimenti viene aggiunto uno 0 (parità pari)
		\begin{itemize}
			\item il numero di bit di parità generato è pari al numero di
				blocchi
			\item tali bit possono essere singolarmente aggiunti di seguito a
				ciascun blocco o tutti insieme in punto precisi delle UI (ad
				esempio alla fine).
		\end{itemize}
	\item Il bit di parità permette di riconoscere errori in numero dispari.
\end{itemize}
Ovviamente questi sistemi hanno un margine di errore, infatti, rilevano bene
tutti gli errori dispari, ma nel caso degli errori peri non li rilevano sempre,
proprio per questo motivo si parla di tolleranza d'errore di un algoritmo di
correzione.
\subsubsection{Esempio}
Possiamo usare il vecchio e classico metodo con il bit di parità a blocchi, in
questo caso utilizziamo quello a blocchi di 8 bit.
\begin{center}
	\begin{tabular}{ll}
		m=10010010&10100011\\
		$m_1$=10010010&$m_2$=10100011\\
		$x_1$=10010010{\bf\color{red}1}&$x_2$=10100011{\bf\color{red}0}\\
		x=10010010{\bf\color{red}1}&10100011{\bf\color{red}0}
	\end{tabular}
\end{center}
Quindi per convenzione quando il messaggio si presenterà in questo modo:
($x=1001001010100011${\bf\color{red}10}). Per convenzione il valori di check
sono collocati nel pacchetto o all'inizio o alla fine (\texttt{tipicamente alla
fine})
\chapter{Reti in area Locale e geografica}

\chapter{Rete internet}

\chapter{Applicazione delle reti}
\printindex
\end{document}
