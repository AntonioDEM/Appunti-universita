\documentclass{book}

\usepackage[utf8]{inputenc}
\usepackage{titlesec}
\usepackage{easylist}
\usepackage{hanging}
\usepackage{hyperref}
\usepackage[a4paper,top=2.0cm,bottom=2.0cm,left=2.0cm,right=3.0cm]{geometry}
\usepackage{blindtext}
\usepackage{tipa}
\usepackage{epigraph}
\usepackage{enumerate}
\usepackage{longtable}
\usepackage{setspace}
\usepackage{verbatim}
\usepackage[T1]{fontenc}
\usepackage{graphicx}
\usepackage[italian]{babel}
\usepackage{amsmath}
\usepackage{pbox}
\usepackage{fancyhdr}
\usepackage{cancel}
\usepackage{tabularx}
\usepackage{booktabs}
\usepackage{multirow}
\usepackage{longtable}
\usepackage{tikz}
\usepackage{tikz-qtree}
\usepackage{subfig}
\usepackage{xcolor}
\usepackage{amssymb}
\usepackage{amsmath}
\usepackage{mathrsfs}
\usepackage{textcomp}
\usepackage{circuitikz}
\usepackage{pifont}
\usepackage{imakeidx}
\usepackage{verbatim}
\usepackage{dsfont}
\usepackage{listings}
\usepackage{color}
\usepackage{upgreek}
\usepackage{tasks}
\usepackage{exsheets}
\usepackage{pgfplots}

\SetupExSheets[question]{type=exam}

\definecolor{mygreen}{rgb}{0,0.6,0}
\definecolor{mygray}{rgb}{0.5,0.5,0.5}
\definecolor{mymauve}{rgb}{0.58,0,0.82}

\lstset{ 
  backgroundcolor=\color{white},   % choose the background color; you must add \usepackage{color} or \usepackage{xcolor}; should come as last argument
  basicstyle=\footnotesize,        % the size of the fonts that are used for the code
  breakatwhitespace=false,         % sets if automatic breaks should only happen at whitespace
  breaklines=true,                 % sets automatic line breaking
  captionpos=b,                    % sets the caption-position to bottom
  commentstyle=\color{mygreen},    % comment style
  deletekeywords={...},            % if you want to delete keywords from the given language
  escapeinside={\%*}{*)},          % if you want to add LaTeX within your code
  extendedchars=true,              % lets you use non-ASCII characters; for 8-bits encodings only, does not work with UTF-8
  firstnumber=1000,                % start line enumeration with line 1000
  frame=single,	                   % adds a frame around the code
  keepspaces=true,                 % keeps spaces in text, useful for keeping indentation of code (possibly needs columns=flexible)
  keywordstyle=\color{blue},       % keyword style
  language=Octave,                 % the language of the code
  morekeywords={*,...},            % if you want to add more keywords to the set
  numbers=left,                    % where to put the line-numbers; possible values are (none, left, right)
  numbersep=5pt,                   % how far the line-numbers are from the code
  numberstyle=\tiny\color{mygray}, % the style that is used for the line-numbers
  rulecolor=\color{black},         % if not set, the frame-color may be changed on line-breaks within not-black text (e.g. comments (green here))
  showspaces=false,                % show spaces everywhere adding particular underscores; it overrides 'showstringspaces'
  showstringspaces=false,          % underline spaces within strings only
  showtabs=false,                  % show tabs within strings adding particular underscores
  stepnumber=2,                    % the step between two line-numbers. If it's 1, each line will be numbered
  stringstyle=\color{mymauve},     % string literal style
  tabsize=2,	                   % sets default tabsize to 2 spaces
  title=\lstname                   % show the filename of files included with \lstinputlisting; also try caption instead of title
}

\linespread{1.2} % l'interlinea

\frenchspacing

\newcommand{\abs}[1]{\lvert#1\rvert}

\usepackage{floatflt,epsfig}

\usepackage{multicol}
\newcommand\yellowbigsqcup[1][\displaystyle]{%
  \fboxrule0pt
  \ifx#1\textstyle\fboxsep-0.6pt\else\fboxsep-1.25pt\fi
  \mathrel{\fcolorbox{white}{yellow}{$#1\bigsqcup$}}}

\title{Appunti di Reti di telecomunicazioni}
\author{Nicola Ferru}
\date{}
\makeindex[columns=3, title=Alphabetical Index, intoc]

\begin{document}
\maketitle
\tableofcontents
\listoftables
\listoffigures
\chapter{Introduzione}
\section{Sommario}
Qui di seguito sono riportati i concetti fondamentali trattati all'interno
del documento
\begin{itemize}
	\item Informazione e segnali;
		\begin{tasks}(2)
			\task L'informazione sussiste solo se il ricevente della
			trasmissione non conosce il contenuto della suddetta;
			\task Per esistere una trasmissione devono esserci:
				\begin{enumerate}
					\item Comunicazione;
					\item mezzo di trasmissione;
					\item informazione.
				\end{enumerate}
		\end{tasks}
	\item Informazioni analogiche e digitali.
		\begin{itemize}
			\item Informazioni analogiche: si dicono grandezze analogiche
				quelle che possono assumere tutti i valori intermedi
				all'interno di un dato intervallo; Si dicono grandezze digitali
				quelle che vengono espresse in modo numerico, senza possibilità
				di discriminare valori intermedi tra due cifre consecutive.\\
				By \href{https://it.wikipedia.org/wiki/Analogico}{Wikipedia}
			\item Informazioni Digitali: Con digitale o numerico, in 
				informatica ed elettronica, ci si riferisce a tutto ciò che
				viene rappresentato con numeri o che opera manipolando numeri,
				contrapposto all'analogico.\\
				By
				\href{https://it.wikipedia.org/wiki/Digitale_(informatica)}{Wikipedia}
		\end{itemize}
\end{itemize}
\texttt{Oggi ormai utilizziamo il digitale perché effettivamente i calcolatori
elettronici gestiscono meglio una codifica rispetto a dei numeri reali. Per di
più costa meno produrre un dispositivo che gestisca segnali digitali rispetto
ad un dispositivo che gestisce mezzi analogici, per esempio la differenza tra
lo standard VHS e lo standard CD/DVD/Blue Ray.}\\
Bisogna anche dire che le trasmissione vengono comunque trasmessi tramite dei
canali fisici ({\bf Analogici}), semplicemente all'interno dei dispositivi che si
occupando i convertire da analogico a digitale e viceversa.
\subsection{Alcune osservazioni}
\begin{itemize}
	\item Non tutte le informazioni costituiscono informazione
		\begin{enumerate}
			\item La notizia comunicata deve per noi essere eclatante;
			\item una persona noiosa non apporta informazione perché ripete
				continuamente gli stessi argomenti.
		\end{enumerate}
	\item Problema di misurazione del contenuto informativo
		\begin{itemize}
			\item {\bf Claude E. Shannon} ({\tt 1916-2001}), fondatore della
				\textit{Teoria Matematica dell'Informazione}, è stato il primo
				ad introdurre la distinzione tra forma e significato nel
				processo comunicativo.
		\end{itemize}
\end{itemize}
\subsubsection{I risultati di Shannon}
\begin{itemize}
	\item Non è possibile definire la quantità di informazione associata ad un
		messaggio già ricevuto, ma piuttosto la quantità di informazione
		associata ad un papabile messaggio
		\begin{itemize}
			\item \textit{``information is that which reduces uncertainty''}
		\end{itemize}
	\item La quantità di informazione associata ad un massaggio è tanto più
		altra quanto più esso è inatteso
		\begin{itemize}
			\item il messaggio ``{\bf domani sorgerà il sole}'' ha un bassissimo
				contenuto informativo perché è assolutamente scontato e banale
			\item il messaggio ``{\bf Domani scoppierà la guerra}'' ha un alto
				contenuto informativo.
		\end{itemize}
\end{itemize}
\section{I segnali}
\begin{itemize}
	\item \textit{Grandezze fisiche variabili nel tempo a cui è associata
		un'informazione};
	\item \textit{L'informazione è associata ad una variazione ({\color{red}
		aleatorio} e non deterministica) della grandezza fisica};
	\item Aleatorio (dal latino ``alea'', gioco di dati) è sinonimo di non
		predicibile a priori (in contrapposizione con deterministico).
\end{itemize}


\chapter{Livelli architetturali bassi}

\section{Gestione degli errori}
Visto che i mezzi fisici possono generare degli errori di trasmissione o
recezione, sono stati inventati dei metodi per riuscire a comprendere se
l'informazione trasmessa sia arrivata a destinazione integra. I due metodi
principali sono:
\begin{enumerate}
	\item Controllo e correzione d'errore;
	\item Recupero d'errore.
\end{enumerate}
\subsection{Rivelazione di errore}
\begin{itemize}
	\item Normalmente si basa sull'aggiunta di ridondanza in trasmissione
		\begin{itemize}
			\item utilizzata in ricezione per rivelare ({\bf ma non correggere})
				gli errori;
			\item la ridondanza richiesta per la rivelazione è molto più
				contenuta rispetto a quella che sarebbe richiesta per la
				correzione (\textit{16-32bit})
		\end{itemize}
	\item Può essere alla base di un'eventuale correzione/recupero
	\item Differenti meccanismi di gestione del codice di rivelazione di errore
		\begin{itemize}
			\item controllo di parità (a blocchi), somma completo a 1
				(\textit{checksum}), etc.
		\end{itemize}
	\item un codice di rivelazione di errore deve rilevare solo modifiche
		casuali.
\end{itemize}
\subsection{Controllo di parità}
\begin{itemize}
	\item Per ogni blocco di bit viene aggiunto un bit pari se il numero di 1
		nel blocco è dispari, altrimenti viene aggiunto uno 0 (parità pari)
		\begin{itemize}
			\item il numero di bit di parità generato è pari al numero di
				blocchi
			\item tali bit possono essere singolarmente aggiunti di seguito a
				ciascun blocco o tutti insieme in punto precisi delle UI (ad
				esempio alla fine).
		\end{itemize}
	\item Il bit di parità permette di riconoscere errori in numero dispari.
\end{itemize}
Ovviamente questi sistemi hanno un margine di errore, infatti, rilevano bene
tutti gli errori dispari, ma nel caso degli errori peri non li rilevano sempre,
proprio per questo motivo si parla di tolleranza d'errore di un algoritmo di
correzione.
\subsubsection{Esempio}
Possiamo usare il vecchio e classico metodo con il bit di parità a blocchi, in
questo caso utilizziamo quello a blocchi di 8 bit.
\begin{center}
	\begin{tabular}{ll}
		m=10010010&10100011\\
		$m_1$=10010010&$m_2$=10100011\\
		$x_1$=10010010{\bf\color{red}1}&$x_2$=10100011{\bf\color{red}0}\\
		x=10010010{\bf\color{red}1}&10100011{\bf\color{red}0}
	\end{tabular}
\end{center}
Quindi per convenzione quando il messaggio si presenterà in questo modo:
($x=1001001010100011${\bf\color{red}10}). Per convenzione il valori di check
sono collocati nel pacchetto o all'inizio o alla fine (\texttt{tipicamente alla
fine})
\chapter{Reti in area Locale e geografica}

\chapter{Rete internet}

\chapter{Applicazione delle reti}
\printindex
\end{document}
