\chapter{Livelli architetturali bassi}

\section{Gestione degli errori}
Visto che i mezzi fisici possono generare degli errori di trasmissione o
recezione, sono stati inventati dei metodi per riuscire a comprendere se
l'informazione trasmessa sia arrivata a destinazione integra. I due metodi
principali sono:
\begin{enumerate}
	\item Controllo e correzione d'errore;
	\item Recupero d'errore.
\end{enumerate}
\subsection{Rivelazione di errore}
\begin{itemize}
	\item Normalmente si basa sull'aggiunta di ridondanza in trasmissione
		\begin{itemize}
			\item utilizzata in ricezione per rivelare ({\bf ma non correggere})
				gli errori;
			\item la ridondanza richiesta per la rivelazione è molto più
				contenuta rispetto a quella che sarebbe richiesta per la
				correzione (\textit{16-32bit})
		\end{itemize}
	\item Può essere alla base di un'eventuale correzione/recupero
	\item Differenti meccanismi di gestione del codice di rivelazione di errore
		\begin{itemize}
			\item controllo di parità (a blocchi), somma completo a 1
				(\textit{checksum}), etc.
		\end{itemize}
	\item un codice di rivelazione di errore deve rilevare solo modifiche
		casuali.
\end{itemize}
\subsection{Controllo di parità}
\begin{itemize}
	\item Per ogni blocco di bit viene aggiunto un bit pari se il numero di 1
		nel blocco è dispari, altrimenti viene aggiunto uno 0 (parità pari)
		\begin{itemize}
			\item il numero di bit di parità generato è pari al numero di
				blocchi
			\item tali bit possono essere singolarmente aggiunti di seguito a
				ciascun blocco o tutti insieme in punto precisi delle UI (ad
				esempio alla fine).
		\end{itemize}
	\item Il bit di parità permette di riconoscere errori in numero dispari.
\end{itemize}
Ovviamente questi sistemi hanno un margine di errore, infatti, rilevano bene
tutti gli errori dispari, ma nel caso degli errori peri non li rilevano sempre,
proprio per questo motivo si parla di tolleranza d'errore di un algoritmo di
correzione.
\subsubsection{Esempio}
Possiamo usare il vecchio e classico metodo con il bit di parità a blocchi, in
questo caso utilizziamo quello a blocchi di 8 bit.
\begin{center}
	\begin{tabular}{ll}
		m=10010010&10100011\\
		$m_1$=10010010&$m_2$=10100011\\
		$x_1$=10010010{\bf\color{red}1}&$x_2$=10100011{\bf\color{red}0}\\
		x=10010010{\bf\color{red}1}&10100011{\bf\color{red}0}
	\end{tabular}
\end{center}
Quindi per convenzione quando il messaggio si presenterà in questo modo:
($x=1001001010100011${\bf\color{red}10}). Per convenzione il valori di check
sono collocati nel pacchetto o all'inizio o alla fine (\texttt{tipicamente alla
fine})
