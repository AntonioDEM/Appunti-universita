\documentclass{book}

\usepackage[utf8]{inputenc}
\usepackage{titlesec}
\usepackage{easylist}
\usepackage{hanging}
\usepackage{hyperref}
\usepackage[a4paper,top=2.0cm,bottom=2.0cm,left=2.0cm,right=3.0cm]{geometry}
\usepackage{blindtext}
\usepackage{tipa}
\usepackage{epigraph}
\usepackage{enumerate}
\usepackage{longtable}
\usepackage{setspace}
\usepackage{verbatim}
\usepackage[T1]{fontenc}
\usepackage{graphicx}
\usepackage[italian]{babel}
\usepackage{amsmath}
\usepackage{pbox}
\usepackage{fancyhdr}
\usepackage{cancel}
\usepackage{tabularx}
\usepackage{booktabs}
\usepackage{multirow}
\usepackage{longtable}
\usepackage{tikz}
\usepackage{tikz-qtree}
\usepackage{subfig}
\usepackage{xcolor}
\usepackage{amssymb}
\usepackage{amsmath}
\usepackage{mathrsfs}
\usepackage{textcomp}
\usepackage{circuitikz}
\usepackage{pifont}
\usepackage{imakeidx}
\usepackage{verbatim}
\usepackage{dsfont}
\usepackage{listings}
\usepackage{color}
\usepackage{upgreek}
\usepackage{tasks}
\usepackage{exsheets}
\usepackage{pgfplots}
\usepackage{amsthm}
\usepackage{chemfig}
\usepackage{verbatim}


\SetupExSheets[question]{type=exam}

\definecolor{mygreen}{rgb}{0,0.6,0}
\definecolor{mygray}{rgb}{0.5,0.5,0.5}
\definecolor{mymauve}{rgb}{0.58,0,0.82}

\lstset{ 
  backgroundcolor=\color{white},   % choose the background color; you must add \usepackage{color} or \usepackage{xcolor}; should come as last argument
  basicstyle=\footnotesize,        % the size of the fonts that are used for the code
  breakatwhitespace=false,         % sets if automatic breaks should only happen at whitespace
  breaklines=true,                 % sets automatic line breaking
  captionpos=b,                    % sets the caption-position to bottom
  commentstyle=\color{mygreen},    % comment style
  deletekeywords={...},            % if you want to delete keywords from the given language
  escapeinside={\%*}{*)},          % if you want to add LaTeX within your code
  extendedchars=true,              % lets you use non-ASCII characters; for 8-bits encodings only, does not work with UTF-8
  firstnumber=1000,                % start line enumeration with line 1000
  frame=single,	                   % adds a frame around the code
  keepspaces=true,                 % keeps spaces in text, useful for keeping indentation of code (possibly needs columns=flexible)
  keywordstyle=\color{blue},       % keyword style
  language=Octave,                 % the language of the code
  morekeywords={*,...},            % if you want to add more keywords to the set
  numbers=left,                    % where to put the line-numbers; possible values are (none, left, right)
  numbersep=5pt,                   % how far the line-numbers are from the code
  numberstyle=\tiny\color{mygray}, % the style that is used for the line-numbers
  rulecolor=\color{black},         % if not set, the frame-color may be changed on line-breaks within not-black text (e.g. comments (green here))
  showspaces=false,                % show spaces everywhere adding particular underscores; it overrides 'showstringspaces'
  showstringspaces=false,          % underline spaces within strings only
  showtabs=false,                  % show tabs within strings adding particular underscores
  stepnumber=2,                    % the step between two line-numbers. If it's 1, each line will be numbered
  stringstyle=\color{mymauve},     % string literal style
  tabsize=2,	                   % sets default tabsize to 2 spaces
  title=\lstname                   % show the filename of files included with \lstinputlisting; also try caption instead of title
}

\linespread{1.2} % l'interlinea

\frenchspacing

\newcommand{\abs}[1]{\lvert#1\rvert}

\usepackage{floatflt,epsfig}

\usepackage{multicol}
\newcommand\yellowbigsqcup[1][\displaystyle]{%
  \fboxrule0pt
  \ifx#1\textstyle\fboxsep-0.6pt\else\fboxsep-1.25pt\fi
  \mathrel{\fcolorbox{white}{yellow}{$#1\bigsqcup$}}}

\title{Appunti di Chimica:\\ Per ingegneria}
\author{Nicola Ferru}
\date{}
\makeindex[columns=3, title=Alphabetical Index, intoc]

\begin{document}
\maketitle
\tableofcontents
\listoftables
\listoffigures
\chapter{Introduzione}
La chimica è la scienza che studia la composizione, la struttura e le trasformazioni della \textit{MATERIA}\\
La Materia
\begin{enumerate}
	\item Composizione (analisi qualitativa e qualitativa)
	\item Struttura-proprietà (es. diamante-grafite)
	\item Modellizzazione e progettazione
\end{enumerate}
Le trasformazione della Materia
\begin{enumerate}
	\item Corrosione (\textbf{es. ferro-ruggine})
	\item Combustione (es. sorgenti di energia)
	\item Sintesi (es. farmaci, pigmenti, nanomateriali, polimeri\dots)
\end{enumerate}
\begin{figure}[h]
	\centering
	\Tree[.Universo [.Energia ciò\ che\ occupa\ spazio\ e\ ha\ massa  ] [.Materia Capacità\ di\ \underline{eseguire un lavoro} ] ]
	\caption{suddivisione tra energia e materia}
	\label{fig:enmat1}
\end{figure}
Un sistema è una porzione delimitata di spazio che rappresenta l’oggetto dello studio mentre l’ambiente è tutto ciò che sta attorno al sistema: l’insieme di sistema e ambiente costituisce l’Universo.
\section{Gli stati della materia}
La materia possiede sostanzialmente tre stati:
\begin{enumerate}
	\item \textit{Solida} - ha una forma definita e un volume proprio;
	\item \textit{Liquido} - ha un volume ma non possiede una forma propria;
	\item \textit{Gas} - non ha né forma, né un volume proprio, si espande in modo da riempire il contenitore che lo contiene.
\end{enumerate}
\section{Proprietà fisiche}
\newtheorem{profisiche}{Definitione}
\begin{profisiche} 
	Proprietà che possono essere osservate e misurate SENZA alterare la composizione della sostanza
\end{profisiche}
\begin{enumerate}
	\item colore;
	\item punto di fusione e di ebollizione;
	\item indice di rifrazione;
	\item densità.
\end{enumerate}
\section{Trasformazioni della materia}
\subsection{Trasformazioni Fisiche}
\newtheorem{trfisiche}{Definizione}
\begin{trfisiche}
	Trasformazioni che avvengono senza alterare la composizione della sostanza
\end{trfisiche}
Esempi di trasformazione fisiche:
\begin{tasks}{2}
	\task ebollizione di un liquido;
	\task fusione di un solido;
	\task sciogliere un solido in un liquido per ottenere una miscela omogenea (ovvero una \textbf{soluzione})
\end{tasks}
\subsection{Trasformazioni Chimiche}
\newtheorem{trchimiche}{Definizione}
\begin{trchimiche}
	Trasformazioni che avvengono \texttt{alterando} la natura delle sostanze coinvolte e portando alla formazione di nuovi composti.
\end{trchimiche}
Un esempio di questo tipo di trasformazione: La combustione del metano. Si parte dal metano e dal ossigeno e si arriva a biossido di carbonio e acqua:
\begin{equation*}
	CH_4+ 2O_2\to CO_2+2H_2O
\end{equation*}
Al termine della trasformazione abbiamo una sostanza differente da quella di partenza, in alcuni casi la procedura non è reversibile.
\section{Sostanza pure}
\newtheorem{sopure}{Definizione}
\begin{sopure}
	Una mataria che ha una composizione omogenea non può essere scomposto tramite una trasformazione fisica in materiali differenti. In quanto non è possibile scomporre ulteriormente la materia
\end{sopure}
\begin{figure}[h]
	\centering
	\Tree [.Può\ essere\ scomposta\ Chimicamente\ in\ sostanze\ più\ semplici? [.Composto ] [.Elemento ] ]
	\caption{Sostanza pura suddivisione}
	\label{fig:Sostanza pura suddivisione}
\end{figure}
\begin{enumerate}
	\item Composto - sostanza formato da almeno due tipi di atomi;
	\item Elemento - tutti gli atomi la costituiscono sono dello stesso tipo.
\end{enumerate}
\begin{table}[htp]
\begin{center}
\begin{tabular}{|c|c|}
	\hline
	Composti&Elemento\\\hline\hline
	Acqua $H_2O$&Ossigeno $O_2$\\
	Anidride carbonica $CO_2$&Diamante $C$\\
	Cloruro di sodio $NaCl$&\\
	Benzene $C_6H_6$&\\
	Etanolo $C_2H_5OH$&\\\hline
\end{tabular}
\end{center}
\caption{Sostanza pura suddivisione}
\label{tab:Sostanza pura suddivisione}
\end{table}%
\section{Miscela}
Composti di Due o più sostanze pure
\chapter{Stechiometria}
\newtheorem{stechiometria}{Stechiometria}
\begin{stechiometria}
La stechiometria è la branca della chimica che studia i rapporti quantitativi (rapporti ponderali) delle sostanze chimiche nelle reazioni chimiche.\\
By \href{https://it.wikipedia.org/wiki/Stechiometria}{Wikipedia}
\end{stechiometria}
Da questa definizione è chiaro che questo sistema verrà utilizzato per una
serie di esercizi potenzialmente presenti all'esame.

\chapter{Modelli atomici}

\chapter{Proprietà periodiche}

\chapter{Soluzioni}

\chapter{Legame chimico}
\section{Introduzione}
In natura le soluzioni costituite da atomi isolati sono rare, di solito, gli
atomi si trovano combinati fra loro per formare dei \underline{Composti}.
Questi possono essere di tre tipi; 
\begin{itemize}
	\item \textit{Molecolare} - si basa sulla condivisione degli elettroni di
		valenza (quelli più esterni) da parte degli che danno origine al
		legamene. La forza che tiene uniti degli atomi deriva dall'attrazione
		che entrambi i nuclei esercitano sugli elettroni condivisi.\\
		\textbf{Esempio:} $H_2,\text{ }O_2 \text{ e } N_2$ dove gli atomi
		mettono in condivisione, rispettivamente, 1, 2 e 3 elettroni di valenza
		ciascuno.
	\item \textit{Ionico} - è dovuto alle forze di attrazione elettrostatica
		che si esercitano tra ioni di carica opposta.\\
		\textbf{Esempio:} $NaCl$ che è formato da cationi $Na^+$ e di anioni
		$Cl^-$.
	\item \textit{Metallico} - gli atomi sono tenuti uniti dagli elettroni di
		valenza che sono liberi di muoversi tra i cationi.\\
		\textbf{Esempio:} Sodio (Na), Oro (Au), Titanio (Ti), \dots
\end{itemize}
\subsection{Teorema di Lewis}
La reattività degli elementi è correlata alla tendenza di raggiungere la
configurazione elettronica del gas nobile più vicino ({\color{red} otteziale} o
doppietto per He). Questa tendenza è nota come \underline{Regola dell'Ottetto}.
\begin{enumerate}
	\item Gli \textbf{elettroni di valenza} giocano un ruolo fondamentale nel
		formare legame chimico;
	\item La condivisione di una o più coppie di elettroni porta alla
		formazione di legami covalenti;
	\item Il trasferimento elettronico da un atomo {\color{red}A} ad uno
		{\color{blue}B} porta al legame ionico.
\end{enumerate}
\begin{equation}
	2Na_{(s)}+Cl_{2(g)}\to 2Na^++2Cl
\end{equation}
Rappresenta il simbolo di elemento circondato da un numero di punti pari al
numero degli elettroni di valenza.
\subsection{Struttura di Lewis}
\begin{table}[h!]
	\begin{tabular}{lcl}
		&Elettroni del core&Elettroni di valenza\\\hline
		Boro&\charge{0=\.,90=\.,180=\.}{B}&$1s^22s^22p^1$\\
		&&Core = [He]\\
		&&valenza = $2s^2\text{ }2p^1$\\
		Bromine&\charge{0=\:,90=\:,180=\.,270=\:}{B}&$1s^22s^2\text{
		}2p^63s^2\text{ }3d^{10}4s^24p^5$\\
		&&core = $[Ar]3d^{10}$\\
		&&valenza = $4s^24p^5$
		\\\hline
	\end{tabular}
	\caption{Comparazione tra elettroni core\\ ed elettroni di valenza}
\end{table}
\charge{0=\.,90=\.,180=\.}{B} $\to$ Simbolo di Lewis
\subsection{Elettronegatività \textit{x}}
Misura empirica della tendenza di un atomo in una molecola ad attrarre gli
elettroni di legame. Secondo \textit{Mulliken} è media dell'\textbf{affinità
elettronica} (tendenza ad attrarre un $e^-$ addizionale) e del
\textbf{potenziale di ionizzazione} (tendenza a mantenere l'$e^-$)
\begin{equation}
	x=\frac{(-\textbf{AE}+\textbf{EI})}{2}
\end{equation}
\textit{Si può prevedere se un legame chimico è ionico o covalente sulla base
della differenza di elettronegatività.}
\subsection{Elettronegatività di Pauling}
In una molecola \textbf{AB} la differenza di elettronegatività tra due atomi
\textbf{A} e \textbf{B} viene determinata sperimentalmente da misure di energia
di legame, facendo riferimento a un valore arbitrario di elettronegatività
assegnato al Fluoro (\textbf{4}).
\begin{tasks}(2)
	\task Legame ionico (\textit{totale trasferimento $e^+$}) $x_A-x_B>2,0$
	$[Na]^+\left[\text{ }\charge{0=\:,90=\:,180=\:, 270=\:}{Cl}\text{ } \right]$
	\task Legame covalente a carattere ionico (\textit{distribuzione carica non
	simmetrica, molecola polare})\\ $0,4\leq x_A-x_B\leq 2,0$
	[\chemfig{H-\charge{90=\:,0=\:, 270=\:}{O}} ]
	\task Legame covalente (\textit{condivisione di $e^-$}) $x_A-x_B<0,4$
	\chemfig{H-H}
\end{tasks}
\subsection{Momento dipolare e polarità Molecole biatomiche}
\begin{tasks}(2)
	\task {\bf Le molecole biatomiche} (\chemfig{H-H}, \chemfig{Cl-Cl}, \dots)
	\textit{omo}-nucleari, ove il baricentro della carica positiva e negativa
	coincide.\\
	\textbf{Esempi:} $H_2, O_2, N_2, \dots$
	\task {\bf Molecole polari} Nelle molecole etero-nucleari la differente
	elettro-negatività degli atomi produce una separazione di carica e quindi
	un dipolo\\
	\textbf{Esempio:} In HCl, Cl ha una frazione di carica negativa ($\delta-$)
	e H ha una frazione di carica ($\delta+$)
\end{tasks}



\printindex
\end{document}
